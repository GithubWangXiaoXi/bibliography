%# -*- coding: utf-8-unix -*-

\chapter{如何避免"破罐子破摔"的心态 - 网友篇}

破罐子破摔是指:罐子已经破了,又往破里摔。比喻有了缺点、错误或受到挫折以后,任其自流,不加改正,或反而有意朝更坏的方向发展。

\section{网友1 - 停止自责,即使止损}

这种现象特别常见。一种是:吃了一口薯片,然后对自己说再吃一片把,然后一片又一片,一不小心一包薯片没了,心里想完蛋了,之前白克制了,心情很不好,心想反正已经吃了,那就明天再重新开始吧,于是又吃了一堆让自己发胖的东西。。。另一种是:期末考试前,本来应该好好复习,但是拿起手机一看就一发不可收拾,最应该做的就是放下手机开始看书,然而,内心开始懊悔,怎么就玩了这么长时间呢,我本来好好学习的,然后意淫一段自己如果把这段时间用来看书了会看了多少,然后再想想别人和我一样在自习室了,人家又干了多少正事,越发痛苦。。。这时候再一看书,还有这么多,压力颇大,本来3个小时看完的东西现在需要1个小时完成,可想而知。

然后我们的开始大脑分泌皮质醇和肾上腺素,深深自责加后悔,但是,却更加不愿意行动,而是对自己说,反正我已经浪费了这么长时间里,那就这样吧。

注意——自责,这种心理不但不会帮助你改变现状,反而会让你心情抑郁,没有足够的心理资源真正用来解决问题。因为你的心智都集中在稀缺的时间上面,不能处理困难的任务,就像穷人钱不够时不能做出正确的选择一样。

我们不愿意去面对自己现在的水平,因为之前的懒惰和放纵,造成目前的局面,之前的逃避让现在的自己需要承受更多的痛苦,于是在焦虑和煎熬中企图挣脱心理上的不适感。通过逃避困难让自己看不见问题,希望问题自动消失。觉得避而不见问题就能自动解决。这是心智不成熟的表现。

真正要做的的是——停止自责,即使止损。不要想自己已经做错了什么,而是从现在开始,珍惜时间克制自己,告诉自己,每个人都没有那么强大的自制力,放纵一下是人之常情\footnote{奶螃蟹 \quad \url{https://www.zhihu.com/question/26822516/answer/81033820}}。

\section{网友2 - 对自身价值的认识}

\subsection{这种心态从何而来?(不考虑极端情况)}

首先,得了解人对自身价值的认识方式。人的心,总是倾向于想外探寻的,并期待从外部得到刺激,尤其是能让自己愉悦的刺激。

同理,人对自身的价值的认识,“通常”情况下,也是从外部获得的。

考高分,获得老师对自己学生价值的认可;

讲道德,获得社会对自己品德价值的认可;

勤工作,获得企业对自己技能价值的认可……

这种价值的外部认可,通常可以决定这个价值的外部价格。\\ \\

其次,得了解人性的贪婪。人,其实总是想着一劳永逸的。而事物是不断发展变化的。所以,这个“一劳永逸”,是有个时间范围的或作用范围的。超过了这个范围,为了继续保持生存,就得学习新的事物,进入新的领域。而这,通常会带来不安全感。

所以,人会更倾向于躲在舒适区,也就是什么都不做。这种倾向的本质,就在于人对于之前付出(一劳)后,对“永逸”这个结果的贪婪。

综上,当人遇到挫折,一般来说,就是自身的价值得不到外部认可。而得不到外部的认可,也就相当于之前的工作的效用,已经超出了效用的时间范围或作用范围。

so,遇挫后的不作为或消沉度日,就相当于躲在舒适区。而这种不作为,在大家看来,就是破罐破摔。

\subsection{如何防止自己进入这种恶性循环?}

从源头做起,就是得重新认识自己,认识自己的价值,认识自己的能力边界。而这3个,用熟悉一点的语言来表达,就是人生观,价值观,世界观。

世界观,也就是世界(社会)的运作方式。你知道自己想要什么(人生观),你自己的能力是什么(价值观),你就会想通过你的能力来获得你想要的。这种付出与收获的关系,未必是1:1的。决定这个投产比的,就是社会的运作方式和你个人能力、目的之间的关系。来个简单的比方。假设社会的运作方式,是水从高往低流,从西向东流。你在东方,想去西方,只能走水路,你的能力是小船和小桨。这时,你的付出大约是100,收获大约是10;而当你的能力是大马力的快艇,这时候,付出大约是100,收获大约是90;而当你能走陆路了,你付出100,你就能得到100;而当你想继续像东走水路,你付出50,也许能得到100;

所以,不同的立场,就有不同的投产比。于是,不同的人,世界观也就不同。而这种不同的世界观,也就形成不同的能力边界。这个能力边界,其实就是付出*投产比。当你知道了你的能力边界,你就不会被你的目的与你的能力之间的那道鸿沟给吓尿。你就会懂得努力提高你的能力边界,甚至你会改变你的目的。这种带有目的性的努力,基本可以根除所谓的破罐子破摔的状态。因为破罐子破摔的人,通常既不了解自己,也不了解社会,基本啥都不会……另外,有个好爹好妈,能来自父母无私的爱的人,基本也不会破罐破摔。因为这种人,他有着最大的安全感。有安全感,就有不断试错的机会。所以,掌握好投胎的姿势和技巧,也是非常重要的\footnote{枯禅 \quad \url{https://www.zhihu.com/question/26822516/answer/34169003}}。
(我不太赞同作者对价值观的理解,价值观应该是人的一种价值判断)

\section{网友3 - 完美主义惹的祸}

大概是完美主义,不能很好,就要很差,我不可以有中间。唉,这个根深蒂固的观念真是深受其害,很浪费时间的。感觉整个人都废了。出了些意外,被直击死穴,就再也起不来了\footnote{\url{https://www.zhihu.com/question/26822516/answer/100888780}}。

\section{网友4 - 自我价值感太低}

之前心理咨询老师告诉我的。

无非是自我价值感太低了(自卑心理),没有树立正确的自我评价体系,完全是围绕别人的意见和看法来构造自己的人生观 价值观。

他这么说:先从简单的开始,给自己的目标要尽可能地实现。比如要养成读书的习惯,今天读1页书,明天两页,这类一定可以完成的要求。先培养起自信,再谈效率啊读书感悟之类高深的问题...

或者前排有人说的做好最坏的打算,来图书馆学习,结果学了一小时玩了三小时手机,那你一开始就做好会玩三小时的准备吧!

亲测这么做比盲目“空想白日梦”的效率高多了。虽然内心仍然时刻憧憬着并坚信我不是普通人,我和他们不一样。我的智商和学习能力一定是非同一般的,我一定是可以学的比别人快效率比别人高,我一天可以学10个小时并且忍住不看手机的...别问我为什么,我总觉得我是绛珠仙子转世,灵与力的化身,我的洪荒之力被久久压抑着,我的身上有五重封印,只是时机未到,无极太子还没有发现我而已!不知道多少人和我一样的想法哦

自己就该是那样的应该是全能的,应该人缘好、长相好、学习考得好,工作找得好,我应该什么都好,但凡有一方面出了差错,我就是个loser。 我应该长相清秀灵动,曲线玲珑,气质优雅,如果我胖 我长相普通,我便不值得爱,不值得拥有幸福。即使有了追求者也会患得患失,我相貌如此泛泛他真的是真心的吗?我应该意志力惊人,能够约束自己,忍受寂寞和孤独,勤恳地耕耘学业,成就一番不俗的事业。种种这些,或许是完美主义吧,又或许只是自我价值感太低了\footnote{Easterrr \quad \url{https://www.zhihu.com/question/26822516/answer/458236023}}。

\section{网友5 - 罐子要及时修补}

破罐子破摔,就词义其实看出一个大概来。

1、罐子破了,遭遇了打击;

2、罐子没法修补,希望木有了;

3、还有再摔的坎坷,没有珍惜了。\\ \\

如何规避?

1、不要抱泥罐子,换一个铁的,意思就是要有足够的能力和实力;

2、罐子破了及时修补,意思是要有身边的良朋益友,给你鼓励和帮助;

3、罐子不要装太多东西,意思是不要有不符合实际的欲望 \footnote{文玩汇 \quad \url{https://www.zhihu.com/question/26822516/answer/129168710}}。

\section{网友6 - 缺乏归属感}
这种心态有一个已成立的假设信念,就是:是个破罐子。已经相信是个破罐子了。不改变信念都是茫然,新摔也只是变个花样摔。

改变信念是根本。怎么改变?认为自己是可以的,接纳自己。虽然自己有局限,有短处,但仍然爱自己,接受自己。无论别人接不接受,爱与不爱。

一层层扒皮来看!

1、为什么会认为是破罐子呢?

因为认为自己不行,或认同别人认为自己不行。~自我否定。自身的短处,若确实不行,那就接受自己不行,确实没刘翔跑的快那就接受,要换个别的比,但最好不要比,每一个人都是独特的,没可比性。请问世界上有谁比你长的更像你呢?不是一个不行就全部都不行。

2、人为什么会否定自己?

因为不接受自己。不接受不完美的自己。那首先要接受人无完人。接受我是独一无二的。

3、人为什么会不接受自己呢?

因为不接受自己这样。哪样?如,不好看,胖,矮,能力低,没本事等等为什么不接受这些所谓的“不好”呢?因为害怕。怕什么?怕自己有这些所谓的不好,别人不接受你。

4、为什么怕别人不接受你?

因为怕别人排除你,不接受意味着排除,否定。意味着你不属于他们,你和他们不是一起的,你不属于他们的系统。意味着你不归属于那里。你害怕在那里没有归属感。

5、为什么害怕没有归属感?

因为你需要归属感。当你不需要归属感的时候,就不会害怕没有归属感。

6、那人为什么需要归属感?

因为有依赖,依赖归属感。

7、为什么会依赖归属感?

当你有归属感,属于那里的时候,才觉得存在,才有存在感。因为从小就依赖于归属感。归属于家庭,归属于生活环境。试想当一个很小的小孩子,连自己生活都不能自理的孩子。不被家庭或外部环境接受的时候会发生什么?(被送走?被抛弃?被伤害?……)对归属感的依赖源自生物的求存本能。

想破\footnote{DUO \url{https://www.zhihu.com/question/26822516/answer/72722875}}。

\section{网友7 - 习得性无助}

为了让你更好地理解”破罐子破摔“的心理学含义,我给你举一个著名的心理实验:

曾有人做过这样一个实验:在一个游泳池里中间放置一块厚厚的玻璃,一边放一头小鲨鱼,一边放一些鲨鱼喜欢吃的小鱼,不给小鲨鱼喂食,它饿得头晕目眩,就冲过去吃小鱼,可是每一次努力都被碰得头破血流而回,它一次次的努力,一次次的无功而返,终于,它绝望了,躺在那里,奄奄一息地饿着。这时候,实验人员将玻璃板抽去,小鲨鱼可以畅通无阻地吃到美味的小鱼了,可是它却再也不愿意去尝试了,因为它知道一尝试就会碰壁。这种现象,我们日常生活称为“破罐子破摔”,心理学家把这种状态叫做“习得性无助”
\footnote{\url{https://www.zhihu.com/question/26822516/answer/34157464}}。

\section{网友8 - 假设漏洞已补上,继续推演}

“为什么会”这个问题排名第一的回答也大致说清楚了,但是从实用主义上来说,偏教科书式了。人很多时候都是纠结的,明知不可为而为之,知道这样干下去不好,但还是会恶性循环。

人之所以会自暴自弃,我觉得应该理解成暂时失去心理动力,无需对这种心态产生焦虑。本质上来说,就是生活中的缺失和挫折引起的。我认为要扭转这种状态,首先还是要从根本上去做。我自己应对困境的方法,就是直面困境本身。比如说我最近生意上碰到了一个难题,一直得不到解决,陷入了僵局。那既然这么久都解决不了,那我就认真想想哪个环节出了问题。如果找出来了,可以肯定就是它了,那就先假设这个问题已经解决了,漏洞补上了。这样我们可以从那种心理状态走出来,把问题再往下推演,这时你会发现原来的那个问题没这么重要了。是的,很多困境只是我们把它当作了困境而已,记住这句话。但是还有小部分是确确实实存在的缺失和困境,我们假装这个漏洞补上后,让这条线活起来先。

对了,你解决问题这个条件是借来的,不是你真正拥有的,那经过这个过程之后你脑子思路也清晰了,也相对理性乐观了,那接下来就会去真正获取这个条件,不要让自己一直陷在这里。这是我自己处理问题,改变心态的一套方法,希望对你有用。

"思考解决问题的时候,如果缺什么牌,那就假设自己已经有了这张牌,然后再推演下去,如果能够成功,那就想办法弄到这张牌"\footnote{西门达达 \quad \url{https://www.zhihu.com/question/26822516/answer/34292366}}。

\section{网友9 - 心理原因:这件事未能如期达成定义为失败}

当我们失误导致事情没有按照预想的发展时,我们预料到结果已无法达到令人满意的程度,可能会出现破罐子破摔的情况。       

选择放弃而不是继续完成剩余部分的心理原因是,直接把这件事未能如期达成定义为失败!定义失败后就意味着这件事没有了意义和价值,于是把剩下没完成的部分定义为无价值,而不值得再去做。       

正确的解法是,考虑一下完全放弃和完成剩余的事情哪件更有价值?明天考试我需要复习4个课件,由于其他原因,我只有复习一个课件的时间,很明显,我明天的考试肯定不能通过。破罐子破摔的心理是,既然明天不能通过考试,那么剩余的时间不如嗨皮一下。

而正确的解法是,我现在有两个选择,一个是嗨皮一下,另一个是选择复习一部分为补考做好准备。这时本来剩余的那一部分,就变得有价值了,而不是被定义为失败,最终从选择上大大降低了破罐子破摔的可能。       

罐子破了,但是我现在好像觉得这个破了的罐子还挺像艺术品,那就摆起来不摔了吧。所以把剩余的破罐子重新赋予价值可能是在心理上解决这一问题的一个途径\footnote{了了 \quad \url{https://www.zhihu.com/question/26822516/answer/799049539}}。

\section{网友10 - 指定可行小目标,一点点找回认可}

消极 自信不足,要要制定可行的小目标,一点点达成,一点点找回被认可的感觉

\section{网友11 - 回本作用和沉没成本}

从行为经济学的角度来看,这就属于回本作用。给你看看下面的吧,以下来自网页。(侵删)

Sunk-cost effect俺意译为回本作用。先介绍一下Sunk cost,中文翻译为沉没成本,就是花出的钱泼出的水,指是没任何捞回本钱指望的东西,钱,时间。这样,从理性的角度,沉积成本不影响你的决策,该咋样就咋样。这世界上大部分人都认为自己是理性的,知道赚一元比赚八毛要多,知道要在有限的时空和腰包里做让自己最爽的事。但是呀,在这沉积成本面前,不知多少人原形毕露。

记得一个例子是某年哈佛MBA的入学招待会,入场当然是个人掏腰包而且价格不菲,结果是人人捞本,从不喝酒的学生那天都干掉好几杯。

举个简单的例子:追女孩子。您老花了不少时间金钱和精力,最后泡上了,要更进一步了(结婚)。您的结婚选择和您花了多少时间金钱和精力没多大关系。(可以返还的贵重礼物当然不在此类)换句话说,您(或者她)决定结婚或者分手不取决于过去发生的这些费用,而取决于其他因素。但是俺们常常听到看到这样的对白:“小w,放手吧,她不适合你。”“狗日的,俺们谈了八年呀!我为她花了好几万,我不能就这样让她离开!”这样的对话,就是经济学家们所讨厌的,卡勒曼他们想听到的。

又有多少人发现自己买了舍不得的吃的菜象龙虾啥的坏了,毅然倒掉的?过食伤胃的道理谁都懂,但是又有几个人吃自助餐是吃饱刚刚好就买单走人的?多少人出国留学读书毕业后发现自己专业难找工作能够毅然转行的? 还有,多少人 开车时走错路能马上掉头的(当然是合乎交规的)?很多人都唧唧歪歪一会后才不情愿的采取行动。

回本作用杀伤力最大的战场是金融股票市场了。说一小点吧。股票证券以及各种金融产品的交易员,如果不能正确认识和采用止损策略,反而执意要弥补无可弥补的损失的话,最终的结果就糟的不能再糟了。从巴林银行的里森到中国航油的陈久霖,同样的开端,同样的轨迹,同样的结局。普通小股民也跑不掉。当股票狂跌时惜售,心里指望哪天能涨回来,结果被牢牢套住,还安慰自己,账面损失不大。这回本作用就是一把不见血的刀,干掉了多少英雄好汉!此外,企业投资大项目通常也没有表现出应有的理性,而常常被回本作用给抓住。有好多研究关于企业投资决策失误的,都是项目开始后发现不对头,但是只能追加投资硬着头皮走到黑了。比如说协和号飞机,开始研究后就发现商业化成本太高,但是没办法,最后飞机还是上天了。回本作用对少数从钓鱼工程中得利的人那简直是福音:这世界上还真有送了一次钱再送一次的“好人”。

个人的回本作用可以用loss aversion 来解释,前面有过解释,这里就不多说了.咱中国人还要另外加上一个反对浪费的理由。企业的回本作用可以用个人问责制度来解释。当决策者面对沉积成本决定撤出时,这就清清楚楚的意味着损失,那么决策者必须为此负责。但是,如果追加投资,嘿嘿,头儿,胜负未定呀,俺们一直在努力。这样决策者还能撑上一阵,也许有翻本的机会也未可知。(当然最后是企业倒了大霉)\footnote{\url{https://www.zhihu.com/question/26822516/answer/129156886}}。

\section{网友12 - 学习牛人,提升自己能力}

因为对自己的不自信和对失败的臣服。

怎么规避的话,我想只有提高能力了,多看看那些比你牛的人,也同时多发现自己的长处,趁着年轻多学东西。

最重要的是明白自己想要怎么样的生活。
努力努力再努力\footnote{\url{https://www.zhihu.com/question/26822516/answer/34155672}}。


