%# -*- coding: utf-8-unix -*-
%%==================================================
\chapter{如何避免让自己成为"思想上的巨人,行动上的矮子" - 网友篇}

当你发现自己也是这个话题-思想上的巨人,行动上的侏儒-的一份子时,其实没必要表露出自责内疚懊恼的情绪,其实有很多人和你一样,面临着这样的困惑。他们和你一样,心底里仰慕着知行统一的伟人(比如列宁,毛泽东),渴望自己有一天也能像他们一样,做一个敢想敢做的人。这一章节主要是让大家了解到对于这个话题,你不是孤独的一份子。

\section{网友1:敢做敢想}

现在的我就处在这种状态当中,或者说我到目前才深刻意识到自己在这条路上越走越远,也把原本的路越走越窄。无论在工作中还是生活中,这样的例子都随处可见。

比如我想发展一种兴趣爱好,几天蹦出一个新想法,然后兴致勃勃去网上搜索方法和途径。结果也只是三分钟热度,从来没有行动过。再比如我明明知道自己不喜欢现在的工作,想要换工作却迟迟行动不起来,就这么一天天得过且过。偶尔意识到自身的问题伤春悲秋一番,却也是毫无用处,接着周而复始。

我会嘲笑自己的软弱,却又像旁人一样对自己恨铁不成钢。久而久之愈发没有自信,不敢做导致想都不敢想了。其实很多时候不是自己的能力问题,而是长时间在这种状态下会让自己失去了开始的勇气。明明很简单的事情也会变得很艰难。

可能我还算不上思想上的巨人。因为这句话应该是要表达有想法有创新。而我那点三分钟热度的小心思,都称不上思想了。但我却是实实在在行动上的矮子,毕业两年了,好像什么也没学到。每个月拿着那点基本工资,偶尔悲伤,慵懒度日。可是天知道我多想改变这种现状啊!20多岁的大好青春就这么白白浪费掉了好可惜,不知道以后的自己会有多厌恶现在的自己。

所以我决定从现在开始每天反省自己,该做想做的事情就立马行动起来。尝试过就算失败也总比什么都不做来的痛快吧!我们明明处在人生最美好的时光,为什么要扭扭捏捏的过并不喜欢的日子呢? 很多事情现在不尝试,以后年纪越大,想做的事情可能也做不来了。与其以后再遗憾,不如现在就行动起来!不知题主是想要解决什么样的问题,不管怎么样,敢想敢做好过只想不做。行动起来吧,拒绝做行动的矮子!!!
\footnote{子非我-我非鱼 \quad \url{https://www.zhihu.com/question/36188944/answer/101152915}}


\section{网友2:眼低手高}

以前我唾弃,现在我接受。行动和思想匹配的人是非常少见的。先做个思想上的巨人想清楚大方向,总比瞎努力好。我们的教育就是让我们变得眼高手低,但这在我眼里是一种褒义词。

因为一般发展:眼低手低—眼高手低—眼高手高对于眼低手高的人来说......emmm.....这种也挺好吧...但对于任何想发展成眼高手高的人来说,眼高手低都是必经之路。
\footnote{本小爷 \quad \url{https://www.zhihu.com/question/36188944/answer/1787086503}}

\section{网友3:自我认知不够}

自我认知不够,想得太多,做的太少。

因为自我认知不够,对自己不够了解,有些时候不知道自己喜欢什么?想要什么?就容易被外界的人事物来干扰内心,做出错误的选择,然后开始尝试,尝试失败,开始自我否定。但不甘心就此,因而尝试第二次,可能又会失败,自我否定又增强,其实说白了,是不了解内心深处的那个自己究竟想要什么,所以才导致了外在的自己做出了错误的选择。

当意识到自己是个思想巨人行动矮子的时候,是一件好的事情,说明内心自我想觉醒,这个时候,如果想做出改变的话,不妨多问问自己想要什么?喜欢什么?尝试去了解自己,认识自己。人只有了解并且认同自己的时候,才会知道什么是好,什么是不好,好与不好并不是外界定义的那些,而是你发自内心的评判。

道可、道非、常道!
\footnote{枚木 \quad \url{https://www.zhihu.com/question/36188944/answer/1341929842}}

\section{网友4:举例说明}

By the way:该网友的观点有点像鸡汤,行文在理,有些观点还是挺新颖的,但有些解释我不太能接受,其中他拿霍金来解释”思想上的巨人,行动上的侏儒”不太合适,因为在我看来,霍金既是思想上的巨人,也是行动上的巨人,不能误以为行动只是人的正常肢体动作,因为除了肢体动作,我们还会说,写,这些基础能力作为人关于思想活动的输出,也可以看做是一种行动。后面作者也强调了刘慈欣关于“三体”的创作,是他作为一位思想上的巨人,也是行动上的巨人的体现。

\subsection{宇宙的起源}

你邻居家的哥哥,每次见到你,就跟你聊,他聊的内容如下:

宇宙的起源,大爆炸到黑洞,如何证明黑洞和大爆炸奇点的不可避免性,黑洞越变越大,但在量子物理框架中,黑洞因辐射而越变越小,大爆炸的奇点不断被量子效应抹平,
整个宇宙空间正是起始于此。

看完这些你什么感觉?

思想上的巨人,行动上的矮子。

大家在看待这句话的时候,不论是看自己还是看他人,常常带有强烈的嘲讽和鄙视。鄙视他人还稍微好点,如果在看待自己的时候,也带有强烈的嘲讽和鄙视,那就糟糕了,
认定自己是思想上的巨人,行动上的矮子,无疑是非常强的负面自我认知。

这句话在我看来,并不值得掺杂情绪,问你个问题:人类历史上,你能想到的最强的思想巨人,行动矮子是谁?  史蒂芬霍金。

他可以算吧,他的头脑,在人类历史上已经留下重要的足迹,说他是巨人不过分吧。他的行动,他的身体基本无法行动,说是矮子不过分吧。

一定有人会杠,说:谁说霍金是行动矮子,他写了《时间简史》,《果壳中的宇宙》,出版了很多书,做了很多学术研究,怎么能说是行动矮子呢?这样说没错,但我问你件事。如果不是霍金,不是大名鼎鼎,如雷贯耳的霍金,而是我开头提到的邻居家哥哥跟你谈论关于宇宙的话题,你会崇拜他么?会觉得他很棒么?多数情况下会觉得他蛇精病吧。

思想上的巨人,本身就是一种重要的行动,霍金写书,把文字“写”下来,只是把头脑中的思想进行记录的过程,而非行动的过程。

真正的行动,是头脑中的思想在发展,在更新,在升级,甚至创造。

一切真正的行动,都始于思想上的行动,所以从这个角度上看,根本不存在思想上的巨人,行动上的矮子这种说法。

我们来到这个世界上,原本的使命很可能千奇百怪,但有多少能够获得父母和周围人的认可和支持,又有多少是一旦表达出来,就会遭到质疑和嘲讽。

你真的确定脑中所想,是自己的真实所想么?如果你也对宇宙感到好奇,被黑洞所吸引,你能允许这些好奇和吸引变成探索么?

想象一下,还是你家邻居,另外一个邻居,他天天白天去上班,晚上回家听说在写小说,你问他写的啥小说,他说是讲宇宙的,是讲外星文明和地球之间发生的故事,
你问外星文明是叫啥?他说:“三体”。你啥感觉?你会崇拜他么?会觉得他很棒么?多数情况下会觉得他蛇精病吧。

说实话,第一次看三体时,有好几个时刻,我都一边看一边问自己,如果是我想到水滴,想到的二向箱,想到的三个太阳升起,外星人迅速脱水,想到汪淼进入神秘的网络游戏《三体》,我会觉得这些点子很棒,并且认认真真的把它们写出来吗?我的答案是不会,我觉得好蛇精病啊。

相关思想会得到发展,探索和壮大吗?

不会的,早就在某个下午小区里散步,刚刚想到时被自我否定了。

如果大刘在头脑中构建了那么庞大的科幻思想体系,却不动笔,会觉得憋得慌吧。如果他头脑中没有构建庞大的科幻思想体系,却要当科幻小说家,估计写几页就写不动了吧。
真的会有思想上的巨人,行动上的矮子吗?

\subsection{思想上是巨人吗}

仔细想,三种情况:

1,思想上如果是巨人,行动不可能是矮子:因为就算其他事情全都不做,满脑子都是思想,却不写出来,很难,会有非常强烈的表达冲动。

2,如果行动上是矮子,思想上十有八九根本就不是巨人:行动一点点之后就做不下去,常常是思想上的枯竭,思想就像源头活水,水不多,流几下就干枯了,后面的行动也会不了了之。

3,思想上确实是巨人,行动上确实是矮子:这种情况你要仔细观察,到底有哪些禁忌在阻碍你的行动和表达。

\subsection{艺术家,新生儿,乔布斯,马斯克}

我常常看到世界级的装置艺术家,用旧纸板做一艘大船,用某种奇葩材料做一个奇葩的装置,你知道杜尚把一个小便池当作艺术品进行展览吗?

如果你想到了同样的念头,会付出行动真的这样做吗?也许你会说,人家是大艺术家,对艺术的理解和诠释,岂是我们普通人能达到的状态。

我想说的是,我在整个生活中,到处看到人们脑中思想的禁忌,这种禁忌与政治无关,而是从小到大不断的被修剪。

人们会用无数种方式打消孩子头脑中稀奇古怪的念头,这些被打消的念头在刚刚冒出来的第一刻,便被扼杀。

当孩子逐渐长大,习惯了外界的对待方式,渐渐内化进心中,便形成了对待自己的方式,当脑中有稀奇古怪念头产生,不用外界动手,自己就进行了否定和扼杀。

于是,一个天生的创作者就此消失。他只能想到主流完全认可的想法和念头。

社会现在对孩子的培养,就像工厂批量生产过程中,对原材料的筛选。所有不符合标准的部分,全部被修剪,规范,扼杀。可是一批批新生儿来到这个世界,真正的意义正是百花齐放,带来无限可能性。成年人应该为扼杀这种可能性感到羞愧。只有极少数幸运儿,他们幸存下来,敢想敢做,就像乔布斯,就像马斯克。你去看看马斯克想的那些奇葩念头,再看看马云当年杭州湖畔花园演讲,像不像传销组织,像不像蛇精病。

据说当年孙正义创立软银,员工就两个,第一天上班,他跳上苹果箱子,慷慨激昂的描绘未来说:“我们十年后会赚到100亿美元。” 第二天两个员工就辞职了。

据说李笑来2013年在北京车库咖啡,讲比特币,闲聊之际,有人问,你觉得比特币还能涨多少?

李笑来说:“100倍”,其他人没说话,默默走开了。我们太容易嘲笑身边那些看起来胡思乱想的人,直到他们的梦想成真。而更多胡思乱想的人,面对的是碰壁,是挫败,是幻想破灭,因为我知道上面提到的例子,都是已经成功的人,都是事后诸葛亮,都是幸存者偏差。

但那又如何?他们确实是幸存者,但他们当年,都是看起来胡思乱想的人中的一个。不要再说思想上的巨人,行动上的矮子,这种话常常带着嘲讽,如果你用这种嘲讽去面对他人,迟早也会被这个世界嘲讽。

在互相嘲讽之间,你我头脑中曾经闪过的小小念头,都被扼杀掉,变成往事成风。留下的只有循规蹈矩的生活和毫无创意的人生。脱口秀大会上,王勉唱到:“世界以痛吻你,你扇他巴掌啊。”

当99年的马云变成了今天的马云,他虽然不说,但是我们都知道,我希望当你的胡思乱想被人嘲讽时,说出那句话:“起开,我是你爸爸。”\footnote{旅行者一号 \quad \url{https://www.zhihu.com/question/36188944/answer/1512550430}}

