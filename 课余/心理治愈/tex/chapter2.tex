%# -*- coding: utf-8-unix -*-
%%==================================================

\chapter{如何避免让自己成为"思想上的巨人,行动上的矮子" - 个人篇}

对于这句话:"思想上的巨人,行动上的侏儒",在我看来,我可以分成两个部分进行解释。

\section{思想上的巨人}

用"巨"字来比喻一个人的思想丰富程度,那怎样的人才算得上思想上的巨人?如果一个人思想活动很丰富,但大脑却始终处于混沌状态,总是看不清某些问题的本质,视野十分狭隘,这样的人在思想上算作巨人吗?我看来,他目前不是,但未来可能是,因为他起码敢去想,敢在大脑混沌的状态中搜寻属于自己的可解释的答案,而有些人却放弃在这种混沌中挣扎,自然而然失去了寻找答案的机会,失去了开阔自己视野的机会。

一个人如果能用逻辑思维能力去解释某种现象,并且能够通过自己的思想去教化指导其他人,那么他就是思想上的巨人。这些思想并不单纯指我们从课本上习得的知识,而是通过自己构建的思维体系抽象化所形成的浓缩思想,它主要体现为人类的智慧,即为哲学。作为思想顶峰上的巨人,他们一般能够用通透的眼光看待现实中不一样但存在一定联系的问题,对于这个世界,这个社会,他们看得很透彻,并且在生活上往往也过得很清澈。

\section{行动上的巨人}

\subsection{哪些行为属于行动}
如果把思想看做是一种意识流,那么行动是意识流的具体输出。思想指导着四肢,脸部的运动,使得我们可以通过各种形式输出自己的思想:四肢的运动是行动,与人说话沟通也是一种行动,把自己的思想用文字记录下来也是一种行动,微表情也可算是一种行动。

其实在我看来,如果一个人单纯有想法却不行动,它的思想就得不到实践的考量,或者不能被具体化,那么在思想的后续发展上很难延续,很难让思想具有连贯性。所以根据木桶效应,要想成为思想上的巨人,行动上至少不能是个侏儒。对于我来说,我觉得自己算是一个思想丰富的人,我善于去记录,去具体化我的思想,或者说对思想形成的意识流进行截流操作,做记录的过程也算是一种行动。

但是对于有计划的具体行动来说,我却名副其实是行动上的侏儒,因为我时常在行动中丧失信心和勇气,一方面是对行动结果的过分期待引起的强迫情绪,另一方面是对行动过于小心翼翼,太过张前顾后,导致对于挺简单的任务迟迟没法按期完成。

\section{思想和行动的内在关系}

这里主要谈一谈在具体计划中,思想和行动的内在关系,这里的行动可以包括利用文字对思想做记录,也就是设计阶段;也可以包括方案实施阶段。

有思想,有目的才能行动,否则只会瞎努力;行动中既要贯彻原有拟订的思路,也要实事求是,随机应变,不能因为发现当下的行动和自己拟订方案不同而丢失信心,变得自暴自弃,甘于现状。

\section{如何避免成为"行动上的矮子"}

每个人都有目标,每个人都有计划,在制定计划的时候,每个人一定会想着能够彻底执行。

目标是计划,执行是手段,只有执行彻底的计划,才算是真正达到目标。

但是目标的设定必须得有点技巧,大目标,或者说总体目标是最终的诉求,但执行过程中不能只盯着大目标,否则执行过程中一遇到困难和挫折,就容易放弃。

应该是先有一个整体规划,然后整体计划的目标进行拆解,拆解成在当前的自身条件下,稍微努力一点,稍微“跳一跳”就可以够得着的目标,执行起来不会有太大的阻碍,也才更容易坚持下去。

目标细分的本质就是,把大目标细分为一个个可量化、可立刻执行出结果的小目标。越具体,越能量化的小目标,越容易实现。

大目标执行结果的关键在于,细分目标+执行有结果。越具体,越能量化的小目标,越容易实现。

无法执行的目标往往有这几个特征:
\begin{enumerate}
    \item 并非自己真心想去做,而是一时兴起或者盲目攀比他人。
    \item 制订了“宇宙级”的目标,完全超出自身实际承受能力,根本实现不了。
    \item 有总体计划目标,但是没有具体实施细则,没有拆解到最小的执行单位。
    \item 贪多嚼不烂,什么都想要,什么都想做,不知取舍,结果是什么都没得到。不走心的计划不是好计划。
\end{enumerate}

这就好比锻炼身体,每个人都知道锻炼对身体素质提升的重要性,尤其是在发生疾病的时候。于是很多人制定了一个“宏伟”的计划:花几千块块钱办一张健身卡,答应自己每天都去健身房,或者每天至少跑5公里,或者每天做50个俯卧撑,可是绝大部分都没能坚持下去,最终因为“今天天气不好”,“今天太累了”,”我今天太忙了“等借口,而导致计划夭折了。在工作学习上也同样如此。

但是如果一开始在做目标拆解细分的时候,不要把要求定那么高,也许执行一段时间后,效果会超乎想象。还是拿锻炼为例,一开始只要求自己每天做5个俯卧撑,或者每天下班后出了地铁站(公交站)跑400米回家,等自己适应后逐渐增加任务量,渐渐地,锻炼便会成为一种习惯,得到的正反馈越多,执行起来就变得容易多了。

如何设定并达成目标,关键在于如何保证自己去执行。目标拆解应该符合SMART原则5要素:具体的(Specific)、可衡量的(Measurable)、可达到的(Attainable)、与目标具有相关性(Relevant)、有明确的截止期限(Time-based)。

要想让目标计划落地为实,真正做到避免出现“思想上的巨人,行动上的矮子”,得真正清楚自己想要什么,然后用一系列可操作性强的方法和工具去踏实地去做,只有这样才不会让目标计划一次次的落空。

让自己把目标量化到每个流程,把目标细化到日常工作、学习和生活中,一步一个脚印达成最终的目标。

只有量化的计划,只有切实可执行,才能达到最终的目标。

不要让自己成为“思想上的巨人,行动上的矮子。”\footnote{如何避免让自己成为“思想上的巨人,行动上的矮子”? \quad \url{https://baijiahao.baidu.com/s?id=1649283506056022858&wfr=spider&for=pc       }}
