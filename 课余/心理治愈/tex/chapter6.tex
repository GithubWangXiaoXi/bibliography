\chapter{充满正能量和充满负能量的人 - 个人篇}

在我看来,正能量是一种短暂性忘却顾虑,它总是抱着某种希望,能够推动人向前的一种“积极“力量,而负能量是一种不断放大风险顾虑,并对其进行思考和分析的“平淡理性“能量。对于正能量和负能量的使用熟练程度因人而异。

不喜欢用正能量,负能量来区分人。因为人的情感包括喜怒哀乐等,十分复杂,并不能完全明确的将其划分成正负两类。那些自以为正能量满满的人,往往是自以为是的; 而负能量的人中往往也有很多是明辨是非的。这两类人并没有严格的好坏区分。正能量可能会做错事,负能量可能会做对事。

正能量的人,他内心往往也隐藏着负能量,负能量的人也向往着拥有正能量。但是对于这两种能量,过尤而不及。对于真正想要成长来说,适当的负能量是必要的土壤; 而适当的正能量是你撒下的种子,它会引导你去克服困难,迎难而上。

这两股能量此消彼长,比如当你消耗着自己的正能量无私帮助他人时,你的负能量也会随之增加(这就是能量守恒)。有些人正能量能够极速增加到一定的水平,但是储存不了多久,消耗得特别快,而转化成的负能量可以储存比较长时间,消耗很慢。

因此如何将两种能量保持在平稳但有一定波动水平上至关重要,有网友建议画画,而我个人感觉冥想是个不错的办法,但是要坚持做下去。因为正能量在被消耗的过程中,会转化成现实中的创造力活动,消耗正能量产生的负能量却被储存起来,不能被消耗,这时就可以通过冥想来让这种能量以蒸发的方式慢慢消散(冥想的时间自己去定义)
