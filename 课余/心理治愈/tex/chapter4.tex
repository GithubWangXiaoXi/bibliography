%# -*- coding: utf-8-unix -*-
%%==================================================
\chapter{如何避免"破罐子破摔"的心态 - 个人篇}

\section{自己做了哪些事属于破罐子破摔}

1、下象棋时有一步走错,整盘棋就不打算下了,莽着让对方吃掉所有的子。

2、写代码时,总是出现bug,一开始有找bug的动力,越到后面,真不想找了,补都不想补了,索性放弃。

\section{原因分析}
破罐子破摔 = 自暴自弃
主要原因:
\begin{enumerate}
    \item 自我价值感很低(看不到自己的优点,时常否定自己,把没有如期完成的事情定义为自己的失败,很在意别人的眼光)
    \item 完美主义者要么把事情做完美,要么把它摧毁(罪魁祸首)。
    \item 如果用经济学的角度来解释这种现象的话:假设我的价值观是珍惜时间,并尽快完美地完成任务,但是一旦我面对挫折无法解决,或者很影响我的情绪时,我会有放弃的心理冲动,一方面是完美主义因素不允许我在困难面前保持软弱,但另一方面是我实在被这个挫折大大影响了心情,导致我工作效率低下,这时我会陷入不断地自责,自责越大,工作效率越低,恶性循环,最终只能放弃(宕机)。
    \item 为什么会有这种恶性循环,其实你想通过投入时间,希望产出的回报能回本,但是你的价值观认为完美的自己才有价值,完全有动力的时候才能创造出价值。你的价值观是二元的,非黑即白的,非完美即失败的,完美才有价值,失败没价值。这样会导致你在回本作用中,一直觉得自己在亏损,能量在不断的消失,无法从套牢的深渊中走出来。
\end{enumerate}

\section{解决方法}
\begin{enumerate}
    \item 减少自责,接受自己
    \item 假设自己能够找到所面临挫折的解决方法,然后接着推演下去,如果后面真的行得通,到后面在来尝试找到当前挫折的解决方法;
    \item 提升自己的能力,让你的能力和你的价值观匹配。
    \item 制定切实可行的,可以量化的计划。每完成一个小计划,就会增加对自己的认可,实现正向循环;而不是在大计划中不断受挫,感受失败,否定自己,这样只会导致恶性循环,找不到自身价值。
    \item 人不能时时保持最佳状态,要接纳那样的自己
\end{enumerate}


