%# -*- coding: utf-8-unix -*-

\chapter{充满正能量和充满负能量的人 - 网友篇}

破罐子破摔是指:罐子已经破了,又往破里摔。比喻有了缺点、错误或受到挫折以后,任其自流,不加改正,或反而有意朝更坏的方向发展。

\section{网友1 - 稳定负能量的人平淡真实}

充满负能量的人,有时他们身上会有一种动人的气质,叫做戏剧感。我时常觉得这种气质是很多用满满正能量武装自己的人所没有的。

唔一好友,我觉得他算是非常稳定的严重抑郁。他从来都淡淡的,不论讲什么样的话,掷地有声或者蜜语甜言,他都来的平平淡淡。但是我们觉得他真,觉得他值得信赖,觉得人生应当就是经历过很多然后回归于平淡就像他那样。很多个夜晚,我们都在和他平静的畅谈中平静了自己。不知你有没有这样的感觉,有时你低落的时候,并不会愿意找一个每天状态都很嗨的人,你还得疲于应付他给你的鸡汤,很聒噪。

还有一个朋友,很高大而好看的一个男生。他有各种奇奇怪怪的性癖。恋母恋脏恋物恋各种能想到的东西,他沉迷和实践于各种性爱情节当中,有些程度真的让我们目瞪口呆。 他说他生活的很累,很不快乐,一直在恐惧变成一种他害怕变成的人。 但是我们其实又太喜欢他了,他头脑里有那么多有趣的想法,相机里拍出了那么多美好的画面,摄影机里创造了那么多神圣的镜头。他风风火火的活着,和内心的矛盾毫无抗争的活着。我在看过他拿给我他自己拍的一个情色片后,想着这些年他的那些样子,觉得所谓醉生梦死就是这么个意思。他孱弱,但其实人要有多坚强,才敢让自己这么接近自己本性的活着。

对于正能量这个词,我从来就不太喜欢,它和乐观与积极向上其实并无直接关联,与什么样的人生态度也挂不上钩。而且我还时常觉得,正能量有时并不是种自然的人生状态。当然这也许和我自己并没有高亢的人生状态有关。我理解的负能量,其实是相对于那种不断追求正确的励志和奋斗精神而言的。要活出那种鸡汤里的人生,我真做不到。我热爱人生,但我从不大声的喊出来,也不会教育身边的人你一定要热爱人生,我就是这种能量状态。因为我深知被理解的不可能,便常常微笑着一言不发,等待着心静如水,也等待笑靥如花,哪有什么波澜不惊?翻江倒海,都留给自己一人承受\footnote{张小玉 \quad \url{https://www.zhihu.com/question/20309007/answer/20676880}}。

\section{网友2 - 正/负能量划分泯灭了正常情绪表达}

大多数情况下,我比较讨厌有人用「正/负能量」这种表达,是因为这两个词泯灭了感恩、祝福、欣慰、嫉妒、自弃、愤怒、悲伤等情绪,简单粗暴的把它们转化一维数轴上的两个方向。

这两个词又阻碍了我们对一个人产生复杂的,客观的积极或消极的评价,让我们对一个人的看法简单的两极化,把人分成两组,一组所谓「正」,一组所谓「负」。更可怕的是说这话的人还沾沾自喜,以为自己使用了贴切精准的表达。他们开始用这种黑白分明的表达给自己、他人、此事、彼端贴上标签,极大的杜绝了对某人某事更深的思考和讨论。

回题目本身。没有人是所谓充满正能量的,没有谁24小时都能开开心心,就我个人经验而言,你眼中「充满正能量」的那种人,可能只是积极的情绪表达的比较外在。再说了,喜欢一个人如斯复杂之事,就别再期望用「正能量」或者「负能量」寻找理由,更妄图用它来思考解决办法。也万万别当出现了让你失望的时刻时,轻轻叹息道,「充满正能量的人果然不会喜欢我们这种充满负能量或是没有那么正能量的人」\footnote{Oulivia \quad \url{https://www.zhihu.com/question/20309007/answer/21718169}}。

\section{网友3 - 正能量爆棚的人往往自以为是}

By The Way,这个网友的回答被群网友diss

我几乎所有的朋友对我的评价都是充满正能量。那我试着回答一下这个问题Y(^_^)Y

负能量的人我是不会喜欢的。我曾经试图拯救过一些对我特别好,但是充满负能量的人。可是最后我发现我真的是太累了。他们就像一个黑洞一样,一点点的拿走我的能量,却一点能量也不能给予我。他们眼里的生活永远是灰色的,和他们聊天就是听他们抱怨。每次和他们谈话过后我真的是只想睡觉,什么力气都没有了(记得有一次在听一个德国朋友抱怨的时候我真的睡着了,醒了以后真的是无比尴尬@.@)。

但是很多时候我也真的很纠结于这个问题。因为负能量的人往往都特别nice,善良,因为他们朋友少嘛,所以只要你抽出一点时间听他们抱怨,他们就会把你当成特别好的朋友,然后对你特别的好。所以虽然很难和他们成为特别好的朋友,但是做普通朋友还是可以的。比如有的时候我觉得自己能量太多了,就可以和他们吃个饭,听听他们生活中的苦恼,然后帮他们排解一下。我觉得人生在世总是善有善报恶有恶报的,在自己有精力的时候去给需要温暖的人一些温暖总是好的。

最后,让我和一个负能量的人在一起?NO WAY。我宁可单身一辈子,呵呵,我还没活够呢,真的不想让他把我对美好生活的向往毁了:)\footnote{Anna \quad \url{https://www.zhihu.com/question/20309007/answer/20636752}}

有些网友开始diss

\subsection{diss环节}

1、正能量爆棚的人往往都自以為是吧。

2、这篇文章充满了对负能量人的偏见。这作者只是个自我感觉良好的所谓正能量者

3、你简简单单把人的情绪摆成正负,感觉很肤浅。你说的那些抱怨都是浅层的,很多人因为情绪无法排解产生的痛苦只是希望理解关注而已。可能是疾病可能是创伤可能你经历的他没有过,你过去了那个坎他没过去。别拿无知的情感理解当作心灵鸡汤。

4、正能量是从负能量转化而来的才强大。

5、你所说的负能量的人,对你好,视你为好友,你却连基本倾听的礼貌都做不到,此为无礼;毫无怜悯同情互助之心,此为不仁;朋友待你为好友,你对他们却不会喜欢,自视为救世主,可怜他们才和他们吃一顿饭,此为不义;对自己的行为毫不自知,没有自知之明,此为不智;大言不惭宣传自己的“光辉事迹”此为不耻。不仁,不义,无礼,无智,无耻。这就是你所说的,“正能量”吗?

\section{网友4 - 负能量的人受正能量的人积极的影响}

忍不住来自己跑来回答这个问题。我男朋友是一个很正能量的人,但是我就很负。所以我总是都很担心他有一天会讨厌我~╮( ̄▽ ̄")╭

但是时间待长之后,会发现,和正能量一起久了,自己也会慢慢的变得正向起来。但愿他会看到这一点~~~

回正题,从我和他的特性来说,他正能量的部分原因是还在读书,对很多事情都想的特别美好,没有什么防备之心,简单来说可以认为,他对风险的把控相对很弱。我负能量的一部分原因是已经毕业工作了,还身处一个复杂的人际圈,尔虞我诈的事情看得多了,特别的谨小慎微。于是,当他面对新的问题和环境时,会来问我意见,希望我能帮他控制风险;当我消极要死,看什么都不顺眼时,我会去挠挠他,让他给点希望和能量~~\footnote{鲁绮莹 \quad \url{https://www.zhihu.com/question/20309007/answer/20675574}}

\section{网友5 - 充满负能量中二的人}

我曾经谈过个充满负能量的男朋友 简而言之对他的回忆就是「不要再让我遇见这傻逼」

楼主知道和一个一天到晚叨逼叨抱怨 说party不好 社会不好 国家不好的中二男生在一起多累吗?自己明明是个脾气暴躁,不学无术的屌丝男,偏偏一副「众人皆醉我独醒」的样子 在这种人身边呆着感受强大的低气压我都快得抑郁症了

过了大概两年终于分手,一下子觉得天都晴了

所以永远不要去喜欢一个眼高手低 只会抱怨的人 多笑笑 对人对己都好\footnote{\url{https://www.zhihu.com/question/20309007/answer/24726226}}

\section{网友6 - 学会分析自己的负能量}

会累,而且很累。我比较积极乐观,遇事看得较开,尽量与人为善。女友因为家庭等因素影响,生性悲观,心胸狭隘(虽然我很爱她,但这是事实)。她几乎每天都要抱怨一些人和事,负能量极重。开始我会认真分析原因认真开导,试图让她乐观豁达,但是,八年过去了,收效甚微,她上班以后,因为工作不尽如她意,负面情绪愈发增强。我越来越觉得累,因为当我看着她“做个正能量的人”的签名,再听着她的各种抱怨时,我知道,任何企图改变人的性格的尝试都是徒然。我还是会和她结婚。

2014年7月21日更新尽管只有12赞同,也是很开心。跟大家汇报一下近况:我惊喜的发现,女朋友慢慢有所改变了!勇敢的辞去了让她抱怨不已的工作,经过半年努力,签下了她现在比较满意的新工作,整个人逐渐理性起来。最重要的是,她能够认真分析那些让她不开心、负能量的事情了,有了判断和取舍,抱怨的事情逐渐少了。我们开始商量结婚的事宜了,开心~\footnote{木二 \quad \url{https://www.zhihu.com/question/20309007/answer/20671224}}

\section{网友7 - 负能量满满的人自救才会被接受}

1、我觉得自己非常想回答这个问题,是因为过去一年来我有经历过类似的困惑。答案如下:那些充满正能量的人,会喜欢一个负能量或是没有那么正能量的人吗? -会的

我身边的女友一半是负能量的人。我喜欢她们的美好,美丽,善良,穿衣服漂亮,都喜欢美剧,等等很多不一而足的可以做朋友的共同点。\\

2、会觉得待在一起累吗? -会的

我和负能量女友在一起时间不能太长,最好不用单独相处,话题不能讲太深。因为我会累。不是我没有试过帮助她们,负能量是一个场,是一种能,不是我摆事实讲道理就能扭转的。如果我一再试图帮助,她们每次都表示我说的对,下次见面她们又陷入更糟糕的境地,那么很遗憾,我会失去继续倾听的耐心。我说的还不是劝人改善职场心态这种事,而是,明明是被炮友拍照应该报警的事情还当成是爱情来纠结。\\

3、如果充满正能量的人没法跟不是正能量的人在一起,那么负能量的人该要怎么办啊? -自救

如果问题里的在一起仅仅是普通朋友的在一起,那么,你只能真的自己想通了,才能强大起来。如果你说的是男女关系,短时间的炮友关系还是可以成立的。如果是严肃的男女关系,要正能量足够强大,并且愿意接受你,而你也愿意改变才可以。我的男朋友也是负能量,但是他愿意为了爱我成为更好的自己。正式交往之前我们已经正视了我们的气场不同,我有几次跟他分手也是这个原因,但是他不肯,一年以来他的情绪控制和人生观都改善了很多,连他公司的同事也发现了他的变化。然而也会有吵架,每次都是因为他负能量辐射或者外溢
\footnote{DaisyGuo \quad \url{https://www.zhihu.com/question/20309007/answer/14702835}}。

\section{网友8 - 充满负能量的人不爱自己,更不会爱别人}

感觉很有资格来答,跟一个充满负能量的人暧昧了五年,半年前才终于看清,毅然放弃,连联系也不想了。

刚认识时他就来表白,我没答应,觉得不熟,后来接触下来,俩人都是有心暧昧,有不开心互相倾诉倾诉,还觉得很贴心。有时说神马,我心情不好,给你打个电话听听你的声音就好了,自己简直美死了。再后来,他有了女友,还是会给我打电话,说他不开心的事,说只有跟我说才好。慢慢的,我发觉不对,他跟我聊天永远都是,他的家里不好,他的同学不好,老师不好,命运不公,生活无趣,前途无望。早期,我还母性大发,觉得他在我面前真实,总陪他安慰他,但是永远换不来一次他变好。后来,有次我也是孤单了,说要不咱们在一起吧,他答应了,两天后,他说他陷入了低落状态,要自己缓缓,后来说要分手。不知什么原因。但我们依然有联系,他仍然每次都抱怨,永远在抱怨。哪怕我有了烦心的事找他说,他还是会拐回自己的烦心事。再后来毕业工作,有时打来电话,仍然再说他的工作不好,心情不好,生活没意思。他又有了女友。我忽然发现,我就是一个垃圾桶,他把所有负面情绪都倒给我,给别人展现出美好积极的自己,我就是有病。再后来,他又分手了,又给我打电话,却借自己喝多,想让我提出复合。我终于看清了,他的负能量什么都不会带给我,我的陪伴不会让他更好,反倒把我自己带入消极状态。终于决定不再跟他纠缠,就像是放过自己。

谈到普通朋友,也认识一个负能量的,女生看不上他,拜金物质!买了房子,我房奴还贷啊!性格不好,工作导致!不会说话,接触不到正常朋友圈!一切因果关系无力吐槽。一个满身负能量的人,不会爱自己,不会爱别人,不会让自己的生活更好,也不会给别人的生活带来更好的可能,除了抱怨不会真正改变什么。所以,能离远点就离远点。朋友不差这一个,何必让自己也闹心\footnote{sunSun \quad \url{https://www.zhihu.com/question/20309007/answer/31479752}}。

\section{网友9 - 理性客观面对生活}

曾经我也是个充满负能量的人。真的,就是从青春期到前几个月的事,持续了七八年了!

当我意识到怀揣负能量永远不能改变命运,还严重影响到身心健康的时候,大学已经是尾声了。但我没有再次悲观,绝望。因为我知道了,只要努力的向上,日子总会一天天好起来!

或许走题了,但我还是忍不住要说。

当我成长以后我开始痛恨这个世界和我所生活的环境。因为从无忧无虑的童年走来我突然接触到了太多的黑暗同时我爷对自己失去了自信和阳光。

我开始沉默,开始抱怨,开始仇恨这个世界的丑恶。

以上充满负能量的经历就不再解释了,总之我感觉经历了一个无爱,迷茫的青春期。

今年暑假我骑行了川藏。但并没有找到所谓人生的觉醒。

回到学校的时候我也就回归了作死的颓废的大学生活。但突然有一天,我意识到:如果继续这样下去,我将一事无成。

是的,我就是这样突厄的转负为正。

时至今日,我早已学会用理性客观的态度面对生活,用积极乐观的姿态迎接挑战。
最后我总结了下我现在的正能量心态:
1:已经这么糟了,我不能让它变得更糟!

2:开心点不会带来成功,也不会避免失败,但它可以让你永远有着生活的希望。

3:不管怎么样,乐观总是强过悲观。

4:负能量者没有任何希望,正能量的活着还有那么一点。

5:就算一无所有,我也可以拥抱明天!

总而言之,我已经变成正能量的人啦!我不鄙视任何曾经和我一样的负能量者!但是我希望和一个充满正能量的人类一起陪伴!\footnote{魏玛斯 \quad \url{https://www.zhihu.com/question/20309007/answer/20639264}}。

\section{网友10 - 真正强大的正能量者}

一般很鄙视负能量者的所谓正能量携带者,都是内心隐藏着负能量,与负能量携带者一起就被诱发出来,所以他们远离负能量者从而避免这种诱发。 

真正强大的正能量者是不会有这种诱发的,你的负能量他都看穿了,他不仅看穿了他还知道从哪里来的,他不仅知道他甚至有办法化解,当然世上如此强大的人极少就是了,而且强大到这个程度的正能量者,要不就是原生家庭对人格塑造得非常好,要补百折不挠从最初的负能量者进化而来\footnote{LIGHTIC \quad \url{https://www.zhihu.com/question/20309007/answer/73010633}}。

\section{网友11 - 圣母般的正能量}

从感情方面来说,这个社会总有人喜欢把自己当作救世主,尤其是很多女生,潜意识里受言情小说影响太严重,自我感觉就是苦情戏女主角圣母玛利亚,然后一个劲儿地去渣男那边找虐,总觉得自己能感化他,他能被自己感动。我们不能绝对地说没有负能量转变为正能量的,但是,有多少?用数学思维来讲,小概率事件就是在一次试验中一般来说是不会发生的,要经历足够多次试验,才有可能发生。那么,生活中,一个正能量的人又有多少正能量去消耗?有那么好运你遇到的那个人就是那个千分之一万分之一被你感化的吗?所以,不要抱有太大希望。

不过,如果给自己的定位正确,或者自认为自己真的是个完全正能量的人,可以无条件无限度地去感化对方,支持对方,包容对方,不求回报,同时心理承受力还要好--因为你所想要给予TA正面能量的人并不一定会接受你的好意,要完全确定能做到这些,那么,真的就是耶稣在世了,我等凡人只能膜拜。不过我相信,正常人都会累的。而且我一向秉持的观点是“神不救不自救之人”。如果一个人他一心消极或者一心浪荡,他没有向好的心,旁人再怎么想帮忙都是没有用的。或者说,从价值观来说,他认为这是适合他的,这就是他的价值观,那么,我们也无权干涉或者批判旁人。\footnote{徐小兔 \quad \url{https://www.zhihu.com/question/20309007/answer/14702147}}。

\section{网友12 - 负能量的人太无私,不喜欢自己}

全无私的那种人类,因为太不合群,最后都飞升到了神界,剩下的都是人不为己天诛地灭的。

所以,负能量人自己都不喜欢自己,凭啥让别人喜欢?更别说还是个正能量型人。

我坚持认为,如果想和什么人在一起,就先让自己变成那样的人才有可能\footnote{\url{https://www.zhihu.com/question/26822516/answer/34155672}}。

\section{网友13 - 负能量的人渴望正能量的人}

作为一个全身心都是满满负能量的人,我想说,是的,不管是正能量还是负能量的人,都不会喜欢一个满心负能量的人。或者说,心里充满正能量的人比较容易得到幸福。

我是典型的心里阴暗外表阳光的负能量过度携带者,对世界的基本假设都是悲观的,所有的内心戏也从不向人倾诉,哪怕平时在一起玩的很疯哪怕一伙朋友是别人眼中的好闺蜜,也从没向她们敞开过心扉。

直到后来遇到一个姑娘,我视她为我最重要的朋友,她是典型的充满正能量的人,让我不由自主的想靠近,想相信。她是我第一个和唯一一个倾诉对象,她改变我很多。可是我没有被改造成一个正能量携带者,相反,最近我的负能量彻底爆棚。原因就是,我把救赎自己寄希望于她,而她帮不了我。我认为她能让我得到救赎,她之于我是最特别的,可是我自己也能感觉到,如果面对一个天天抱怨人生悲观消极的人太久了,都会心力交瘁,所以当她对我有一丝冷淡的时候,当她陪其他朋友的时候,当她和男朋友打很长时间电话的时候,我的烦躁都会加剧。当她说「还是得靠你自己想明白自己调整好状态」的时候我就觉得她对我不耐烦了不想管我了,可是天知道她说的对啊!确实应该这样。

喏,看到了吧,像我这样总是怀着这种心理的人能被那些充满正能量的人喜欢麽?ps.哈哈哈要是我的朋友们看到我这段回答会是怎样惊恐的表情啊,这和他们眼中的我完全两个人-_-# \footnote{\url{https://www.zhihu.com/question/20309007/answer/20678208}}

\subsection{有网友提出妙招 - 关心事实,别太关注自己}

这种阴郁的情况,感觉如果不是受到重创,就多半是自我意识过剩。怎么理解“自我意识过剩”呢?“过剩”是超出需求,“自我意识”是意识到(区别于与他人的)自己的存在;简单来说,就是太关注自己。不信看看日记,是不是多数是自己的感受,好些句子以“我”来开头?

而解决这个问题,就要将多余的意识转移出去,这意味着:关心事实!

比如尝试用文字、画作或声音描述所见、所闻,甚至所想(所谓“心理素描”);又如记录自然笔记、读书笔记。

