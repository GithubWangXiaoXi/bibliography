%# -*- coding: utf-8-unix -*-
%%==================================================
\chapter{语言与文化}

\section{英汉对比研究}

翻译教学和研究的经验表明:翻译理论和技巧必须建立在不同语言和文化的对比分析的基础上。英汉互译的几项基本原则和技巧,如选词(diction)、转换(conversion)、增补(addition)、省略(omission)、重复(repetition)、替代(substitution)、变换(variation)、倒置(inversion)、反说(negation),拆离(division)、缀合(combination)、阐释(annotation)、浓缩(condensation)、重组(reconstruction),以及时态、语态、语气、习语、术语等的译法,都集中地体现了英汉的不同特点。机器翻译是让计算机按照人所制定的程序和指令进行不同语言的对比转换,也离不开对比分析。翻译之所以困难,归根结底是因为语言差异和文化差异。因此,对比、分析和
归纳这些差异,便是翻译学的重要任务。

不同语言的对比分析不仅有利于教学和翻译,也有助于语言交际。通过对比分析,人们可以进步认识外语和母语的特性,在进行交际时,能够有意识地注意不同语言各自的表现方法,以顺应这些差异,防止表达错误,避免运用失当,从而达到交际的目的。

\subsection{英汉语言对比研究}

英汉语言对比研究侧重于英汉的共时对比分析研究,旨在寻找、描述并解释英汉的异同,尤其不同之处(difference)和特殊之处(peculiarity),并将研究的结果应用于有关领域。

英汉语言方面的对比研究可以从各种角度进行,方法也应该多种多样,如:浅层、中层和深层的层次对比、认知一功能语法对比、专题对比等等,从微观、宏观、微观与宏观相结合、理论、应用、理论与应用相结合等各个方面进行,其研究的结果也会多种多样,如:彼此相同,同中有异;彼此相异,异中有同;似同实异,似异实同;此有彼无,此多彼少,此浅彼深,此宽彼窄,此囊彼贬,此可彼否,此简彼繁,此显彼隐,此刚彼柔,此聚彼散,此形被意,此物彼人,此明彼暗,此代彼复,此实彼虚,此动彼静,此客彼主,此迁彼直,如此等等,或者相反的情况。英汉对比分析的角度举例如下。

1、层次对比分析

1)表层:结构、形式、语义对比分析,如
\begin{enumerate}[label=\circled{\arabic*}]
    \item 语音:英语的语调,汉语的声调、音位(数目、划分、英汉元音与辅音的近似与空缺)、词与句的重音、英汉语的节奏
    \item 文字:字母与汉字、象形表意文字、表音文字、表意文字、形声文字
    \item 词语:构词法、形态特征、词的界定、虚词、新词、外来词、习语
    \item 语义:语义场、词化程度、词汇的理据性、词语搭配、英汉词语的语义关系,如:对应、近似、并行、包孕、交叉、替换、空缺、冲突;一系列微妙的差异、如:贬裹、宽窄、新旧、感情色彩、民族风韵、国情特色、语用背景
    \item 句法:句子成分、句子结构、句型、语序
    \item 篇章:篇章粘连性(cohesion)、篇章连贯性(coherence)、语句的逻辑扩展模式、篇章结构。
\end{enumerate}

2)中层:表达方式、方法对比分析

如涉及语法、修辞、逻辑、语体、语用等方面的表达方式方法对比分析;在特定语境中,用词、造句、成章、衔接、搭配、表现风格等方面的表达方式对比分析,如:英汉的刚性与柔性、形合与意合、繁复与简短、物称与人称、被动与主动、静态与动态、抽象与具体、间接与直接、替换与重复等。

3)深层:中西文化和思维方式差异在英汉语言的表现(这一层次的对比分析已进入下面所列的宏观的语言与文化的研究范畴)

如:西方的理性思维与中国的悟性思维是英语与汉语的哲学背景。这一深层差异必然表现在用词造句成章的各个方面,如:英语较常受亚里上多德的演绎法逻辑思维模式的影响,常用“突显”语序;常用形合法、结构被动式和概括笼统的抽象性词语:注重显性街接、语法关系和语义逻辑,注重形式接应,“前呼后应”;喜欢词语和结构的主从分明、长短交错和替代变换;表达方式比较严谨、精确,模糊性较小,歧义现象较少,用词造句遵守严格的词法和句法,造句成章也服从某种逻辑规则,适合于科学思维和理性思维。

汉语常用意合法、意念被动句和生动具体的形象性词语;常采用非演绎式的、往往是领悟式的归纳型、经验式的临摹型或螺旋式、漫谈式的思维模式,注重时间先后和事理顺序,常用“自然”语序;注重隐性连贯,较常只把事情或意思排列起来,让读者自己去领悟其间的关系;注重语流的整体感,喜欢词语和结构的整体匀称、成双成对、对偶排比和同义反复,表达方式注重整体性,较多依赖语境;汉人习惯于整体领悟,常常通过语感、语境、悟性和“约定俗成”来表达和理解语句。

2、认知-功能语法对比分析(cognitive-functional approach)

运用这一对比分析法,以功能为主,论述英语和汉语如何表达同一认知范畴(如:指称、方所、时间、比较、数量、正反、关系),所采用的方法和手段有何异同,研究的路子是“由里到外”,如赵世开主编的《汉英对比语法论集》(上海外语教育出版社,1999)就是这一研究方法的代表作。

3、专题对比分析

王宗炎(1985)认为,“对比分析可分为两种:一种是纯理论研究,没有其他目的。这种分析应当力求详尽、明白(explicit)、严格,像其他纯理论研究一样。可是这是长期性工作,要用大量的人力、物力和时间。一种是供教学用的,只要抓住要点,加以阐述,不要求细大不捐,系统也不必十分严密。在我国目前情况下,后者的可行
性无疑较大。”按照后者的构想,有的学者打破以上层次界限,用宏观和微观相结合的方法,就英汉的语法特征、修辞手段、表现方法、思维习惯、文化因素等方面,取其具有普遍性、代表性、典型性和实用性的专题,深入对比分析。这种“专题对比分析法”,不求系统齐全,而求突出特征差异,可以深入研讨而避免泛而不深。这类专题如:
\begin{enumerate}[label=\circled{\arabic*}]
    \item 英汉某项语法特点比较
    \item 英汉语序特点比较
    \item 英汉某项修辞特点比较
    \item 英汉某类语篇特点比较
    \item 英汉某类语体特点比较
    \item 英汉某项语用特点比较
    \item 英汉成语特点比较
    \item 英汉当代某种文风比较
    \item 造成英汉语言差异的语言内部因素
    \item 造成英汉语言差异的语言外部因素
    \item 造成Chinglish的语言因素与文化因素
\end{enumerate}

\subsection{英汉语言文化对比研究}

\subsubsection{语言学研究从微观到宏观的历程}

长期以来,语言学者关心的是语言的内部形式和结构,但却忽视与语言密切相关的社会和文化因素。现在人们逐步认识到,掌握一门语言,至少应掌握两套规则:一是结构规则,即语言的内部系统,包括语音、词汇、语法等等;二是使用规则,即使用语言是否恰当、得体并为人接受的种种因素。一句话,若语音、词汇和语法都正确,但不合说话人的身份,用于不恰当的场合,或不问听话人的社会和文化背景,违反当地的风俗习惯,不合所在国的国情,仍然达不到交际的目的,甚至会产生严重后果。这些使用规则,均涉及社会和文化的种种因素。于是,专门研究这类问题的学科也就诞生了,其中“社会语言学”、“跨文化交际学”和“文化语言学”就是日益受到重视的学科。

我国有关语言与文化的研究,虽然早已有学者论述,但长期以来未得到足够的重视。1950年,罗常培发表了《语言与文化》,篇幅不长,却涉及古今中外的语言与文化,可谓中国文化语言学的先驱之作。他在该书中强调,“语言学研究不能抱残守缺地局限在语言本身的资料以内,必须要扩大研究范围,让语言现象跟其他社会现象和意识联系起来,才能格外发挥语言的功能,阐扬语言学的原理。”该书从语词的语源和变迁看过去文化的遗迹,从造词心理看民族的文化程度,从借字看文化的接触,从地名看民族迁徒的踪迹,从姓氏别号看民族来源和宗教信仰,从亲属称谓看婚姻制度。全书从语词的涵义论述语言与文化的关系,对我国的社会语言学和文化语言学研究具有开创意义。

中国文化语言学的诞生,是有一定的原因的。自从1898年马建忠借鉴拉丁语法发表了我国第一部汉语语法《马氏文通》以来,汉语学界注重的是汉语语法研究,尤其是模仿西方早期的语法理论,引进西方语法学的概念、术语和方法,着重对具体的语言现象作分析与描写,没有形成完整的汉语语法体系。汉语不同于西方语言,但汉语研究却老是模仿西方的语言理论。用西方的语言理论来描写汉语,往往陷入重重困惑。20世纪80年代中期以后,我国语言学界受到国外社会语言学和跨文化交际学研究潮流的冲击,越来越多的学者深感只有把语言与社会文化结合起来,把语言作为一种社会现象和文化现象来研究,以文化闸释语言,以语言阐释文化,并研究这两者的关系,才能更正确、更全面、更深刻地认识语言;只有把中国文化引入中国语言的研究领域,才能使逐渐成为冷门的中国语言学从西方的理论框框里摆脱出来,建立起符合汉语特点的中国语言学理论。

语言学者还发现,外国的语言学研究已经走过了相当一段历程,进入了一个新大地。现代语言学的奠基人索绪尔(F.de Saussure)于1916年所提出的语言学说,虽然是“哥白尼式的革命”,对语言学的贡献很大,但他的学说只是“就语言和为语言而研究语言”,不研究与语言密切相关的外部要素。后来“美国语言学的牛顿”布龙菲尔德(L.Bloomfield)于1933年所提出的结构主义描写语言学(descriptive linguistics)理论,只注重指写、分析和归纳语言的表层现象,还是停留在语言的内部结构。接下去看姆斯基(N.Chomsky) 在1957年发起的语言学革命(Chomskyan Revolution中提出了转换生成语法(Transformational-Generative Grammar),研究实验室里“理想的说话人和听话人”的语法能力,研究如何用有限的规则去生成无限的句子。

遗憾的是,所有这些都还是语言的微观研究,局限于语言本身,且其对象一般不超过句子的范围。然而,人不是生活在实验室的真空管里。人不仅要自己说话,而且要与人交谈,不仅要说单个的句子,而且要把句子组成话语来进行交际,于是产生了篇章语言学(text linguistics)。语言学家还要研究“谁在何时用何种语言向谁说话”(Who speaks what language to whom and when), "什么人在什么情况下为了什么目的对什么人说什么话并且得到什么结果”,“什么时候说话,什么时候不说话,跟堆说什么,什么时候说,在什么场合说,用什么方式说”,怎么说才会让人接受,才算得体、恰当,才能产生预想的结果。这就产生了一门新的学科,着重研究社会上各种人的语言交际能力,叫做社会语言学(sociolinguistics),那是20世纪60年代的时候,吕叔湘称之为“语言学的又次解放”。但是,这还不够。人不仅要生活在同个社会里与本国人、本族人交谈,还要到外国去,到异乡去,与不同文化背景的外国人、外族人交际,这就不仅有语言不同的问题,还有文化不同的问题,于是又产生了门新学科,叫做跨文化交际学(intercultural communication)。这门学科也诞生于60年代,专门研究不同文化背景的人进行交际所产生的言语问题和非言语问题。

从上述可以看出,国外语言学研究走的是一条由语言内部扩展到语言外部、由微观扩展到宏观、由小范围扩展到人范围,“由小店铺扩展到百货商店再扩展到超级市场”的历程。

\subsubsection{英汉文化语言学的研究对象}

英汉文化语言学的研究对象可以分为理论研究、实用研究和应用研究:

1、理论研究。包括英汉文化语言学的体系、性质、任务、目标、原则、方法;英汉语言与中西文化的关系、英汉文化语言学与相关学科(如人类语言学、社会语言学、交际语言学、民族语言学)的关系。从语种和文化类型的角度出发,还可分为汉语文化语言学和英语文化语言学。

2、实用研究。侧重点是从中西文化的异同解释英汉语言的异同,或从英汉语言的异同探究其文化渊源,突出差异之处。由于研究的范围从英汉语言扩展至中西文化,进入了宏观领域,按照不同的目的、角度和范围,实用研究可以多种多样、丰富多彩,以下仅举几例:

1)体现中西文化差异的英汉词语涵义。例如,英汉成语、谚语、俚语、敬语、谦语、俗语、熟语、委婉语、禁忌语、交际语、问候语、礼貌语、祝福语、致谢语、吉祥语、恭维语、称谓语、歇后语、双关语、体态语、拟声词、重叠词、颜色词、数量词、动物词、植物词、食物词、味觉词、季节词、方位词、服饰词、人体器官词、自然气象词、政治词语、宗教词语、角色用语、民间信仰词语,含有典故和神话的词语、民族或国情惯用词语、新生事物词语以及其他词语所包含的不同文化涵义。

英汉词义的对应关系是一个复杂的课题,涉及不同的语言、社会、文化等种种因素。英汉词义的贬裹、宽窄、新旧、感情色彩、民族风韵、国情特色、语用背景等·系列微妙的差异以及英汉词义之间的对应、近似、并行、包孕、交叉、替换、空缺、冲突等种种关系往往在双语词典中得不到反映,缺乏外语语感和有关知识的学生常常难以把握语义、语境和语用所涉及的种种问题而导致望文生义,张冠李戴、汉式英语和欧化汉语。

2)体现中西思维方式差异的英汉语法特征。例如,西方的理性思维在英语语法中所体现的“法治”,重形合,重形式接应(formal cohcsion),结构要求齐整,句子以形寓意,以法摄神,句段严密规范,采用焦点句法,因而“语法是硬的,没有弹性”且富于强制性:中国的悟性思维在汉语语法中所体现的“人治”,重意合,重意念连贯(semantic coherence),结构不求齐整,句子以意役形、以神统法,句段流泻铺排,采用散点句法,因而“语法是软的,富于弹性”和灵活性。

3)体现中西文化差异的思维方式及英汉表现方法。西方文化的科学精神与中国文化的人文精神体现于不同的思维方式,如个体分析与整体综合、逻辑抽象与直觉形象、三段推理与辩证统一、客体思维与主体思维、空间观念与时间顺序、线性演绎与螺旋归纳、精确性与模糊性等。这一层次的对比分析与社会、文化、心理、语境(实际情景、上下文、篇章语境)相结合,已进入宏观的文化语言学范畴。从中西文化差异(如生产方式、生活习惯、历史传统、哲学思想,价值观念、民族心理、审美情趣等方面)解释中西不同的思维方式及其在汉英语言里不同的表现方法,例如,体现西方科学认知型文化的逻辑理性、客体思辨和抽象思维在英语的表现之一是:追求名词化、物称化和抽象化,崇尚客观、间接、规范和错落、变换、替代的表达法;体现中国政治伦理型文化的中庸和谐、阴阳平衡和辩证统一在汉语的表现之一是:追求均衡美和对称美,喜用对偶、排比、重
复、重叠的表达法,以及常用同义组合和反义合成的四字格。

\section{翻译理论}

翻译是一门交叉学科和开放性学科。翻译不仅是语言的问题,而且还受文化的影响和制约。翻译的难易和优劣,与语言有关,更与文化有关。我国著名翻译家王佐良(1989)主张把翻译研究与文化比较结合起来,把文化比较看作是翻译研究中的决定因素。他指出,“翻译不仅涉及语言问题,也涉及文化问题。译者不仅要了解外国的文化,还要深入了解自己民族的文化。不仅如此,还要不断地把两种文化加以比较,因为真正的对等应该是在各自文化中的含义、作用、范围、感情色彩、影响等等都是相当的。翻译者必须是一个真正意义上的文化人。人们会说:他必须掌握两种语言,确实如此,但是不了解语言当中的社会文化,谁也无法真正掌握语言。”当代符号学翻译观把译论的基础由对比语言学扩展到对比文化学的广阔天地。建立文化翻译学,通过社会文化对比来解决翻译中语言之外的各种问题,这已日益受到学者们的关注。

当代某些翻译理论和标准大多过分强调忠实于原文,忽略了翻译的其他因素,尤其是文化差异、翻译动机、译文用途和译文读者。中国对外宜传类题材的汉译英就是一个突出的例子。从事这项工作的老专家发现许多“译文无误,但外国读者费解。其‘病因’在于中外文化(广义的文化)差异甚大,我国大陆的情况尤其特殊。本来,一般外国人对中国事物就知之不多(正如一般中国人对外国事物也知之不多一样)。再加上四十年来社会制度和意识形态的巨变,影响到人陆生活的每一个方面,语言文字作为信息、思想、观念和感情交流的符号,随之也发生了巨大的变化。大陆上大量独特的词汇和说法,在外语中难以找到现成的词汇来表达。”(段连城,1992)对于这类题材的英译,首先必须有很强的跨文化意识和对外宣传意识,充分认识到中西文化的差异,把握原文的意图,预测译文读者可能的反应,注重宣传效果,采用“解释性翻译”,才能避免汉式英语所产生的
“莫名其妙”、“不知所云”、“哭笑不得”、“各取其意”的效果。

对于英汉互译来说,E.Nida的“对等反应”理论应该是具有不同文化背景的原作读者与译作读者的等效反应。他写道:“传统翻译理论把翻译的重点放在语言的表现形式上,新的翻译理论则认为,翻译的重点不应当是语言的表现形式,而应当是读者对译文的反应,还应把这种反应和原作读者对原文所可能产生的反应作对比。”这里所说的原作读者和译作读者反应的对比,其实已是文化翻译学的问题了,也就是他及许多学者所指出的,要格外重视两种不同文化的差异,力争翻译上的同等效果。译者应该精通语言和文化,做一个人类文化学的语言专家。

\section{语言与文化的关系}

关于语言与文化的关系。英汉文化语言学研究英汉语言与中西文化的相互关系,用中西文化解释英汉语言,或用英汉语言解释中西文化。英语语言学、汉语语言学、中国文化学、西方文化学以及其他相关学科是研究的基础。本学科的研究除了必须其备普通语言学、理论语言学等基本理论知识之外,还必须着重从文化的角度去研究语言的内部系统,
如语音与文化、词语与文化、语义与文化、语法与文化、修辞与文化、语篇与文化、语体与文化、语用与文化,文风与文化等。词语是文化的积淀,最能明显地反映文化的特征。从文化方面研究语义,可以建立一门新的学科——文化语义学。语篇是表达和翻译的基本单位,应该作为研究语言的突破口。语篇与文化的关系,思维方式如何表现于语篇,都是重要的课题。

\subsection{语言文化的相互关系}

从宏观的角度看,语言与文化互相影响,互相制约。语言是文化的载体,文化是语言的“管轨”,这已成为许多学者的共识。然而,从微观的角度分析,在寻找、描写和解释具体的语言现象或因素与具体的文化现象或因素的相互关系时,却出现许多复杂的情况。大体说来,这种相互关系至少有以下三类:互相关联;互不关联;部分关联。

1、互相关联:某种文化因素基本上可以解释某种语言现象,或某种语言现象基本上可以解释某种文化因素。例如,用中国人的“悟性”解释汉语的“意合”,用西方人的“理性”解释英语的“形合”;用中国的“主体意识”解释汉语的“人称”,用西方人的“客体意识”解释英语的“物称”;用汉语的“形象性”解释中国人的“意象思维”,用英语的“抽象性”解释西方人的“逻辑思维”,如此等等。英汉语言与中西文化互相关联的现象正是本学科研究的重点。

2、互不关联:某种语言现象没有与之相关联的文化因素,两者无法互相解释。这方面常有一些“似是而非”的对应关系。王宗炎教授(1997:148-154)曾指出一些令人信服的例证,例如,英语的单数第·人称“i”处处大写,这不是说英语的人“个人主义和自我存在的象征,带着膨胀了的个人主义色彩”,而是如The World Book Dictionary(1981:1044)所解释的:"In the old handwritten manuscripts a small i was likely to be lost or to get attached to a neighboring word , and a capital helped kcep it a distinct word " .又如,汉语常用“我们”代替“我”,如“我们认为”、“在我们看来”,其实是“我认为”、“在我看来”的意思,这并不是因为“汉人重社会,表现出集体主义倾向”,所以才“乐意使用集体的字眼”。用“我们”代替“我”,并非汉语独有的特点,英语也常用we代替I,这种用法在一般英语词典就可查到,许多英语惯用法词典还列举用we代替I 的多种用法,如:Inclusive we(说话者把听话者包括在内),Editorial we(表示作者代表编辑部),Paternal we(带有关怀意味),Royal we(用作君主自称,相当于中国皇帝自称的“朕”)等。用we而不用I显得比较谦逊,许多人在一般的、轻松的文字或讲话中也常用we而不用I。再如,汉语喜用缩略语,这不能解释为“汉人注重勤俭节约”,因为缩略语是一种普遍的语言现象,英语也有大量的缩略语,那又如何解释呢?

如此看来,英汉语都有类似的用法,又找不出其不同文化酒义的依据,就不能按主观的推想理所当然地解释英语或汉语用法特有的文化因素了。又如,西方人自我介绍时先说名字,再说单位,汉人自我介绍时先说单位,再说名字,这并不是因为“西方人以自我为中心”,而“汉人重社会,表现出集体主义倾向”。西方人自我介绍时先说职称或尊称,再说名字;汉人自我介绍时先说名字,再说职称或尊称,这也并不是因为“西方人看重通过竞争得到的荣誉和地位”,而汉人不看重所得到的荣誉和地位。这类词序的问题,不能从上述的文化因素去找原因,因为其显而易见的漏洞是:如果不是自我介绍,而是介绍他人或不同文化背景的外国人,上述的英汉语词序的文化涵义又如何解释呢?

3、部分关联:某种语言现象中只有部分可以找到与之相关联的文化因素,或某种文化因素中只有部分可以找到与之相关联的语言现象,关联中有不关联的部分,或不关联中有关联的部分,两者无法完全互相解释,必须加以分别说明。论证常常只考虑到问题的一方面而忽略了另一方面,因而经不起反证。例如,汉民族追求和谐的心理常常表现在汉语词语结构排列的匀称性,因而四音节成了汉人喜闻乐见的语音结构,这是有道理的但为了强调这一点,接着说三音节的词语儿乎都是贬义的,这就片面了因为三音节词语中只有部分是贬义的,如:“愉阴风”、“点鬼火”、“抓辫子”、“戴高帽”、“拍马屁”、“走后门”等,其他的就不一定,甚至可以是裹义的,如:“爱国家”、“有理想”、“学先进”、
“反贪污”、“顾大局”、“攀高峰”等。这个结论应把“几乎都是”改成“部分”,因而很难解释汉语三音节结构的文化涵义。

又如,汉民族的文化传统提倡“尊卑有序”、“长幼有序”、“内外有别”、“亲疏有别”,这种等级和亲疏观念常常表现在汉语词语结构排列的次序,如“君臣”、“男女”、“父子”,“婆媳”、“中外”等,这种解释是有一定道理的,但也应说明还有其他的格式,如:“阴阳”、“雄雄”,“叔伯”、“敌我”等,这类词语的顺序就难以用上述的文化涵义来解释
了。英汉语言与中西文化互相关联的现象常常有例外的情况,互相关联的现象还常常是相对、相比较而言的,在互相解释的时候应予以充分的注意,以确保论证的严密性和完整性。

\subsection{语言文化相关联分析时的注意点}

在解释和论证英汉语言与中西文化的相互关联时,还应注意避免以下情况:结论正确,例证不妥或引申有误;例证正确,引申不妥或结论有误;顾此失彼,互相矛盾;“以偏概全”或“以全代偏”,把个性当作共性或把共性当作个性。例如,英语rice可以概括汉语的“米、稻、谷、米饭”,poetry可以概括汉语的“诗、词”,temple可以概括汉语的“庙
宇、寺院、圣堂、神殿、教堂”、cousin可以概括汉语的“堂兄弟、堂姐妹、表兄弟、表姐妹”,spirit可以概括汉语的“精神、心灵、灵魂、鬼怪、妖精、魔鬼”,此外,汉语表达烹饪方法的词语比英语多,分类较细,如此等等,从而证明英语词语的概括性比汉语强,并与英美民族的抽象性思维和汉民族的具象性思维联系起来。

这样的解释值得商椎,因为这一论证经不起反证,汉语也可以用“桌子”概括英语的table,desk,用“虾”概括英语的shrimp,prawn,lobster;用“会议”概括英语的meeting,
conference,council,congress,convention,convocation等;用“神”概括英语的god,spirit,deity,divinity等;用“意见”概括英语的contrary opinion,different views , complaints , reservation , objection , dissatisfaction , criticism ,disagreement等,这又如何解释呢?有一种解释是:这类同样语义范围内词项数量的多少及复杂性与其文化的注重点、着重点、常用性或重要性(cultural emphasis)等因素成正比:词项较多,切分较细,表明其所指对象在该文化中较为注重或常用,但不与上述的文化因素联系起来解释。

在寻找语言特征或文化特征时,个性与共性、部分现象与普遍现象的界定有个量化和定性的问题。如果大量的调查研究证明,某种语言现象或文化现象拥有绝大多数的例证,这些例证又足以证明在该语言中或该文化中的本质特征,并经得起例外的反证,这就可以确定其为特征或特点,否则就难以定论。语言与文化所表现的现象十分复杂,涉及的知识非常广泛,研究其特征及相互关联时应作具体、动态、多角度的分析,并用大量可靠的例证加以论证,不能轻易一概而论,不能为了得出某种预想的结论而采取“一刀切”的态度、忽视可能还有反面的例证。