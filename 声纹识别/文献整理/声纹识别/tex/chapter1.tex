%# -*- coding: utf-8-unix -*-
%%==================================================
\chapter{这是什么?}

\href{https://blog.csdn.net/qq_34218078/article/details/104454388}{Generalized end-to-end loss for speaker verification}

\href{https://blog.csdn.net/qq_26369907/article/details/90760456}{TE2E和GE2E损失函数区别}

这是$\mathbb{ Q}$-book \LaTeX 书籍模板,当前版本为 \version 。

这份模板主要基于上海交大的学位论文模板
\footnote{\url{https://github.com/sjtug/SJTUThesis}}修改得到
,结合少量个人审美喜好,重新定制了定义、定理等环境。由于这个模板本是用于一项书籍翻译计划,因此其中的一些环境,例如“观察”、“规则”、“关键点”等,读者可能用不到,可以根据自己的需求适当修改。

你也可以通过邮箱\texttt{jey74165@163.com}给我发邮件反映遇到的问题。不过作者水平有限,或许有些问题也无法解答,还请见谅。

\section{文档说明}

\subsection{准备工作}

要想灵活使用、魔改这个模板来撰写自己的书籍,需要对\emph{TeX系统}有一定的了解,也需要掌握基本的\emph{TeX技能}。

\begin{itemize}[noitemsep,topsep=0pt,parsep=0pt,partopsep=0pt]
	\item {\TeX}系统:所使用的{\TeX}系统要支持 \XeTeX 引擎,且带有ctex 2.x宏包。一般来说,只要安装了的\emph{完整}TeXLive或MacTeX发行版就不会出现问题。
	\item TeX技能:尽管提供了对模板的必要说明,但这并不是一份“ \LaTeX 入门文档”。用户应当有一定的\LaTeX 使用经验。
\end{itemize}

\subsection{字体与选项}

$\mathbb{ Q }$-book暂不提供可选项,直接用命令\verb|\documentclass{qbook}|载入即可。如有需要,用户可以根据自己的需求进行相关的添加或修改。

本模板所使用的字体仅为宋体,{\kaishu{楷体}}和{\heiti{黑体}}等自带字体,用户不会在字体问题上折腾太多精力。

\subsection{编译方式}

最简单的办法是直接双击模板文件夹中的compile.bat文件,在命令行模式下编译;当然使用你配置好的Tex编辑器也是可以的。编译失败时,可以尝试手动逐次编译,定位故障。
\begin{lstlisting}[basicstyle=\small\ttfamily, caption={手动逐次编译}, numbers=none]
xelatex -no-pdf qbook
biber --debug qbook
xelatex qbook
xelatex qbook
\end{lstlisting}

\section{模板文件介绍}

本节介绍$\mathbb{ Q }$-book模板中主要文件和目录的功能。

\subsection{格式控制文件}

格式控制文件控制着书籍的表现形式,包括以下两个文件
\begin{itemize}[noitemsep,topsep=0pt,parsep=0pt,partopsep=0pt]
	\item qbook.cfg
	\item qbook.cls
\end{itemize}
其中,“cfg”和“cls”为文件格式。

\subsection{主文件}

主文件qbook.tex的作用就是将你分散在多个文件中的章节重新“拼合”成一本完整的书。
当我们用\LaTeX 写书时,肯定不希望一直在同一页面码字,那样会显得非常臃肿,而且不便于以后的修改和查找。所以在使用本模板的时候,你的章节内容和素材会被“拆散”为各个部分,例如前言、概览、各章节及参考文献等。
在qbook.tex中通过\verb|\include{xxx}|命令将各个部分包含进来,从而形成一本结构完整的书籍。

\subsection{各部分源文件}

被“拆散”的各个部分的源文件存放于tex文件夹中,是论文的主体,以“章”为单位划分,其中包括:

\begin{itemize}[noitemsep,topsep=0pt,parsep=0pt,partopsep=0pt]
	\item cover.tex——用于绘制封面。
	\item preface.tex——前言。
	\item overview.tex——概览。
	\item chapter(xxx).tex——各章主体内容。
	\item 参考文献列表由bibtex插入,不作为一个单独的文件。
\end{itemize}

\subsection{图片存放}

figure文件夹放置了需要插入文档中的图片文件(支持PNG/JPG/PDF/EPS格式的图片)。
在qbook.cls中已经使用\verb|\graphicspath|命令定义了图片存储的顶层目录,所以在插入图片时,图片路径的顶层目录名“figure”可省略。

\subsection{参考文献数据库}

目前参考文件数据库目录只存放一个参考文件数据库qbook.bib。
关于参考文献引用,可参考第\ref{chap2}章中的例子。
