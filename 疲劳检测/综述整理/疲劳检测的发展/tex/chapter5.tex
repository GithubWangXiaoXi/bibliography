%# -*- coding: utf-8-unix -*-
%%==================================================
chap5\chapter{疲劳检测相关专业术语}

\section{信号处理知识}

\subsection{共振峰}

\subsection{倒谱}

参考\href{https://baike.baidu.com/item/%E5%80%92%E8%B0%B1/9851556?fr=aladdin}{百度百科:倒谱}

倒谱(cepstrum)一种信号的傅里叶变换谱经对数运算后再进行的傅里叶反变换。由于一般傅里叶谱是复数谱,因而又称复倒谱。


\section{医学知识}

\subsection{交感神经和副交感神经}
LF,HF,LF/HF均衡性

\subsection{纺锤波}
EGG不同信号频段的纺锤波

\subsection{运动伪影}

除BPF和ASF外,最小均方(LMS)的自适应滤波器还能有效地减少在BVP信号\cite{liu2018self}中的运动伪影,LMS滤波器的一个局限性是它对输入信号的缩放很敏感。

运动伪影是磁共振成像最大的问题之一 ,在许多磁共振检查中运动伪影不可避免的随机噪声更明显地降低图像质量 ,它使图像模糊并且沿相位编码方向产生鬼影运动包括各种生理运动和身体运动。\footnote{\url{https://zhidao.baidu.com/question/1545558467918962427.html}}
伪影(Artifacts):是由于设备或病人造成的。
是指原本被扫描物体并不存在而在图像上却出现的各种形态的影像。伪影大致分为与患者有关和与机器有关的两类。
CT图像伪影指图像上与实际解剖结构不相符的密度异常变化,它涉及CT机部件故障、校准不够及算法误差甚至错误等项目,要消除此类伪影,需根据图像伪影的形状、密度变化值及扫描参数等进行具体问题具体分析。第三代CT机的图像伪影具有一定的普遍性,又特别以环状伪影为最常见。
分类有:
\begin{itemize}
    \item 运动伪影
    \item 混淆伪影或包裹伪影
    \item 化学移位伪影
    \item 化学性配准不良伪影
    \item 截断伪影
    \item 磁敏感伪影
    \item 拉链伪影
    \item 交叉激励
\end{itemize}

MR图像的运动伪影通常是指由于受检者的宏观运动引起的伪影。这些运动可以是自主运动如肢体运动,吞咽等,也可以是非自主运动如心跳,血管搏动等。运动可以是随机的如胃肠蠕动,吞咽等,也可以是周期性运动如心跳和血管搏动等
\footnote{MR运动伪影 \quad \url{http://www.360doc.com/content/15/1207/09/15645340_518464634.shtml}}。

运动伪影出现的原因主要是由于在MR信号采集过程中,运动器官在每一次激发,编码及信号采集时所处的位置或形态发生了变化,因此将出现相位的偏移,在傅里叶转换时会把这种相位的偏移误当成相位编码方向的位置信息,把组织的信号配置到一个错误的位置上,从而出现运动伪影。

运动伪影具有以下特点:
\begin{itemize}
    \item 主要出现在相位编码方向上;
    \item 伪影的强度取决于运动结构的信号强度,后者信号强度越高,相应的伪影越明显;
    \item 伪影复制的数目,位置受基本正弦运动的相对强度,TR,NEX,FOV等因素的影响。
\end{itemize}

随机自主运动伪影是指在不具有周期性且受检者能够自主控制的运动造成的伪影,如吞咽,眼球转动,肢体运动等造成的伪影。随机自主运动伪影的特点:
\begin{itemize}
    \item 主要造成图像迷糊;
    \item 伪影出现在相位编码方向;
    \item 受检者可以控制。
\end{itemize}

\begin{figure}[t]
\centering
    \includegraphics[width=5in]{example/test.pdf}
    \caption{EEG信号.}
\end{figure}

\subsection{公式排版}

这里有举一个长公式排版的例子,来自\href{http://www.tex.ac.uk/tex-archive/info/math/voss/mathmode/Mathmode.pdf}{《Math mode》}:

\begin {multline}
\frac {1}{2}\Delta (f_{ij}f^{ij})=
2\left (\sum _{i<j}\chi _{ij}(\sigma _{i}-
\sigma _{j}) ^{2}+ f^{ij}\nabla _{j}\nabla _{i}(\Delta f)+\right .\\
\left .+\nabla _{k}f_{ij}\nabla ^{k}f^{ij}+
f^{ij}f^{k}\left [2\nabla _{i}R_{jk}-
\nabla _{k}R_{ij}\right ]\vphantom {\sum _{i<j}}\right )
\end{multline}

\subsection{SI单位}

使用\verb+siunitx+宏包可以方便地输入SI单位制单位,例如\verb+\SI{5}{\um}+可以得到\SI{5}{\um}。

\subsection{定理环境}

在这个模板中,定义了如下几个环境
remark(注),mythm(定理),myprop(性质),mydef(定义),example(例)。
amsmath还提供了一个proof(证明)的环境。
我们举例说明它们的用法。

注环境
\begin{remark}
	存在事先给定的一系列基本操作,并且这些基本操作永远不会改变。
\end{remark}
\begin{remark}
	每个操作都可逆。
	\label{o1.2}
\end{remark}
\begin{remark}
	每一个操作都是确定性的。
\end{remark}
\begin{remark}
	各个操作可以按任何顺序组合。
\end{remark}

性质环境
\begin{myprop}{}{}
	存在一些预先定义的永不发生改变的作用(action)。
\end{myprop}

\begin{myprop}{}{}
	每一个作用都可逆。
\end{myprop}

\begin{myprop}{}{}
	每个作用都是确定性的。
\end{myprop}

\begin{myprop}{}{}
	任意的一系列连续的作用仍然是一个作用。
\end{myprop}

例子环境
\begin{example}
	天地玄黄,宇宙洪荒。
	\soln
	
	日月盈仄,辰宿列张。
\end{example}

定义环境
\begin{mydef}{域}{1}
	设$S$为一个非空集合,其上有“加法”(记作$+$)与“乘法”(记作$\cdot$)两种代数运算. 若满足以下条件,则称$(S,+,\cdot)$ 构成一个域(field).
	\begin{itemize}
		\item[(1)] $(S,+)$构成一个交换群.
		\item[(2)] 若记$S^{*}=S-\{0\}$,其中$0$为群$(S,+)$中的单位元,则$(S^{*},\cdot)$也构成一个交换群.
		\item[(3)] 乘法对加法有分配律:$a ( b + c ) = a b + a c$.
	\end{itemize}
\end{mydef}

关键点环境
\begin{keypoint}
	伽罗瓦理论在分析从有理数域$\mathbb{ Q }$扩张到新的域的运算或操作时很有用。我们的大问题可以用伽罗瓦理论来回答,数学中其他的一些历史问题也同样可以用伽罗瓦理论来解答。
\end{keypoint}

定理环境
\begin{mythm}{望远镜公式}{2}
	$\left[\mathbb{Q}(a, b) : \mathbb{Q}\right]=\left[\mathbb{Q}(a, b) : \mathbb{Q}(a)\right]\left[\mathbb{Q}(a) : \mathbb{Q}\right] $
\end{mythm}

\begin{proof}
	
	\rthm{thm:2}告诉我们,对任意$s\in S$,均有$\lvert Orb(s)\rvert \cdot \lvert Stab(s)\rvert=\lvert G\rvert=p$. 于是$\lvert Orb(s)\rvert $ 整除$p$,这里$p$是一个素数。从而$\lvert Orb(s)\rvert $等于1或$p$,也就是说,\textbf{所有轨道的大小要么为1,要么为$p$}. 于是整个集合$S$就被划分为两部分,一部分是大小为1 的轨道,另一部分是大小为$p$的轨道,如图9.4所示。
	
	假设大小为1的轨道有$m$个,大小为$p$的轨道有$n$个,则有
 \begin{equation}
		m+p\cdot n=\lvert S\rvert
 \end{equation}
	注意到\rdef{def:1},\textbf{那些$\lvert Orb(s)\rvert =1$的元素$s$即为稳定元},这就表明有$m$个稳定元。从上式立刻看出$\lvert S \rvert \equiv  m\; (\bmod\; p)$.	
\end{proof}

\section{表格}

这一节给出的是一些表格的例子,如表\ref{tab1}所示。

\begin{table}[!hpb]
	\centering
	\bicaption[指向一个表格的表目录索引]
	{一个颇为标准的三线表格\footnotemark[1]}
	{A Table}
	\label{tab1}
	\begin{tabular}{@{}llr@{}} \toprule
		\multicolumn{2}{c}{Item} \\ \cmidrule(r){1-2}
		Animal & Description & Price (\$)\\ \midrule
		Gnat & per gram & 13.65 \\
		& each & 0.01 \\
		Gnu & stuffed & 92.50 \\
		Emu & stuffed & 33.33 \\
		Armadillo & frozen & 8.99 \\ \bottomrule
	\end{tabular}
\end{table}
\footnotetext[1]{这个例子来自\href{http://www.ctan.org/tex-archive/macros/latex/contrib/booktabs/booktabs.pdf}{《Publication quality tables in LATEX》}(booktabs 宏包的文档)。这也是一个在表格中使用脚注的例子,请留意与threeparttable实现的效果有何不同。}

下面一个是一个更复杂的表格,用threeparttable实现带有脚注的表格,如表\ref{tab2}。

\begin{table}[!htpb]
	\bicaption[出现在表目录的标题]
	{一个带有脚注的表格的例子}
	{A Table with footnotes}
	\label{tab2}
	\centering
	\begin{threeparttable}[b]
		\begin{tabular}{ccd{4}cccc}
			\toprule
			\multirow{2}{6mm}{total}&\multicolumn{2}{c}{20\tnote{1}} & \multicolumn{2}{c}{40} &  \multicolumn{2}{c}{60}\\
			\cmidrule(lr){2-3}\cmidrule(lr){4-5}\cmidrule(lr){6-7}
			&www & \multicolumn{1}{c}{k} & www & k & www & k \\ % 使用说明符 d 的列会自动进入数学模式,使用 \multicolumn 对文字表头做特殊处理
			\midrule
			&$\underset{(2.12)}{4.22}$ & 120.0140\tnote{2} & 333.15 & 0.0411 & 444.99 & 0.1387 \\
			&168.6123 & 10.86 & 255.37 & 0.0353 & 376.14 & 0.1058 \\
			&6.761    & 0.007 & 235.37 & 0.0267 & 348.66 & 0.1010 \\
			\bottomrule
		\end{tabular}
		\begin{tablenotes}
			\item [1] the first note.% or \item [a]
			\item [2] the second note.% or \item [b]
		\end{tablenotes}
	\end{threeparttable}
\end{table}

\section{插入图片}

\XeTeX 可以很方便地插入PDF、PNG、JPG格式的图片。插入PNG/JPG的例子如\ref{fig1}所示。
这两个水平并列放置的图共享一个“图标题”(table caption),没有各自的小标题。

\begin{figure}[!htp]
\centering
\includegraphics[width=4cm]{example/by-nc.png}
\hspace{1cm}
\includegraphics[width=4cm]{example/gzh.jpg}
\bicaption{中文题图}
{English caption}
\label{fig1}
\end{figure}

这里还有插入EPS图像和PDF图像的例子,如图\ref{fig2}和图\ref{fig3}。这里将EPS和PDF图片作为子图插入,每个子图有自己的小标题。子图标题使用subcaption宏包添加。

\begin{figure}[!htp]
\centering
\subcaptionbox{EPS 图像\label{fig2}}[3cm] %标题的长度,超过则会换行,如下一个小图。
{\includegraphics[height=2.5cm]{example/m2.pdf}}
\hspace{4em}
\subcaptionbox{PDF 图像,注意这个图略矮些。如果标题很长的话,它会自动换行\label{fig3}}
{	\includegraphics[scale=0.5]{example/figep.eps}}
\bicaption{插入eps和pdf的例子(使用 subcaptionbox 方式)}{An EPS and PDF demo with subcaptionbox}
\label{fig4}
\end{figure}




\section{插入代码}

这里给一个使用listings宏包插入源代码的例子:
\begin{lstlisting}[language={C}, caption={一段C源代码}]
#include <stdio.h>
#include <unistd.h>
#include <sys/types.h>
#include <sys/wait.h>

int main() {
pid_t pid;

switch ((pid = fork())) {
case -1:
printf("fork failed\n");
break;
case 0:
/* child calls exec */
execl("/bin/ls", "ls", "-l", (char*)0);
printf("execl failed\n");
break;
default:
/* parent uses wait to suspend execution until child finishes */
wait((int*)0);
printf("is completed\n");
break;
}

return 0;
}
\end{lstlisting}


\section{参考文献管理}
\label{sec2.5}
\LaTeX 具有将参考文献内容和表现形式分开管理的能力,涉及三个要素:参考文献数据库、参考文献引用格式、在正文中引用参考文献。
这样的流程需要多次编译:
\begin{enumerate}[noitemsep,topsep=0pt,parsep=0pt,partopsep=0pt]
\item 用户将论文中需要引用的参考文献条目,录入纯文本数据库文件(bib文件)。
\item 调用xelatex对论文模板做第一次编译,扫描文中引用的参考文献,生成参考文献入口文件(aux)文件。
\item 调用bibtex,以参考文献格式和入口文件为输入,生成格式化以后的参考文献条目文件(bib)。
\item 再次调用xelatex编译模板,将格式化以后的参考文献条目插入正文。
\end{enumerate}

参考文献数据库(thesis.bib)的条目,可以从Google Scholar搜索引擎\footnote{\url{https://scholar.google.com}}、CiteSeerX搜索引擎\footnote{\url{http://citeseerx.ist.psu.edu}}中查找,文献管理软件Papers\footnote{\url{http://papersapp.com}}、Mendeley\footnote{\url{http://www.mendeley.com}}、JabRef\footnote{\url{http://jabref.sourceforge.net}} 也能够输出条目信息。

下面是在Google Scholar上搜索到的一条文献信息,格式是纯文本:

\begin{lstlisting}[caption={从Google Scholar找到的参考文献条目}, label=googlescholar, escapeinside="", numbers=none]
@phdthesis{"白2008信用风险传染模型和信用衍生品的定价",
title={"信用风险传染模型和信用衍生品的定价"},
author={"白云芬"},
year={2008},
school={"上海交通大学"}
}
\end{lstlisting}

推荐修改后在bib文件中的内容为:

\begin{lstlisting}[caption={修改后的参考文献条目}, label=itemok, escapeinside="", numbers=none]
@phdthesis{bai2008,
title={"信用风险传染模型和信用衍生品的定价"},
author={"白云芬"},
date={2008},
address={"上海"},
school={"上海交通大学"}
}
\end{lstlisting}

参考文献的引用:
\begin{itemize}
\item 参考文献在正文中被引用,使用命令\verb+\cite{key}+,如\cite{M91}。
\item 参考文献未引用但仍希望列在书末的参考文献中,使用命令\verb+\nocite{key}+,如\verb+\nocite{WI64,G03,D01,JS03}+.
\end{itemize}
\nocite{WI64,G03,D01,JS03}
